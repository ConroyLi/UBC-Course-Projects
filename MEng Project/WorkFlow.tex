\documentclass[12pt]{article} % The document class with options

\usepackage[margin=1in]{geometry}
\usepackage[utf8]{inputenc} 
\geometry{a4paper}
\usepackage{newtxtext,newtxmath}
\usepackage[T1]{fontenc}
\usepackage{amsmath}
\usepackage{amsfonts}
\usepackage{microtype}
\usepackage{graphicx}
\usepackage{listings} % For formatting and highlighting code
\usepackage{color}    % For colors in code highlighting

\begin{document}
\setlength{\parskip}{1em} 
\setlength{\parindent}{0pt}
\newcommand{\vect}[1]{\mathbf{#1}}

\begin{titlepage}  % This starts a title page environment
    \centering    % Center everything on the page

    %--- Add space at the top of the page ---
    \vspace*{2cm}
    
    %--- Title ---
    \normalsize \textbf{MEng Project Log} \\
    \vspace{0.5cm}  % Space between lines
    %\normalsize\textbf{Comparison of Gauss-Newton Method with Marquardt Modification with \\
    % Nelder-Mead Algorithm for Parameter Estimation in Algebraic Models} \\
    \vspace{2cm}  % Space between the title and the author name
    
    %--- Author ---
    \normalsize by\\
    \vspace{1cm}
    \normalsize Jincong Li \\ 
    \vspace{1cm}
    \normalsize M.Eng, The University of British Columbia, 2024
    \vspace{11cm}  % Space between the author and the date
    
    %--- Date ---
    \normalsize 8th May, 2024

    \vfill  % Push the following content to the bottom of the page
    %--- Bottom part of the page ---
    © Jincong Li, 2024
\end{titlepage}
\tableofcontents
\newpage
\section{8th May}
\begin{enumerate}
    \item Applied for the access to ICICS 227
    \item Applied for the access to compute canada
\end{enumerate}

\section{9th May}
\begin{enumerate}
    \item Installed Gmesh, Paraview and get access to compute canada
    \item Read Eigenfrequency analysis from COSMOL 
    \item Learn how to use Gmesh: geometry part and the mesh part with two case: vertical plane and cylinder
    \item Install the Simflow
\end{enumerate}
\section{10th May}
\subsection{Important Functions}
\begin{enumerate}
    \item simflow : run simflow to solve
    \item mpirun -nx simflowOmpi -npx
    \item simGmshCnvt -msh *.msh :mesh
    \item simPlt -type vtk : post processing
    \item gmsh -3 : convert geo to msh
    \item cd ~(home) .(current) ..(previous)
    \item vi :q :w :q! : quit, write, write\&quit, and quit without saving
    \item cp (source) (destination) :copy
    \item mv (source) (destination) :move
    \item scp :copy from others computer
    \item rm -r(folder)
    \item scp -r ineogi@beluga.computecanada.ca:~/scratch/CavityTutorial .
    \item scp conroyli@beluga.computecanada.ca:~/scratch/CavityTutorial/debug1/*.vtk .
\end{enumerate}
\subsection{Case 1: Lid Driven Cavity}
\begin{enumerate}
    \item 10 time steps with 0.1s
    \item saved in `debug1`'
    \item which simflow :give the location of the first version of simflow
    \item /simflow-Nihar/bin/simflow
    \item vi simflow.config
    \item simPlt -type vtk -min 0 -last 10 in CavityTutorial
\end{enumerate}
    \subsection{Files}
    \begin{enumerate} 
        \item cavity.geo
        \item cavity.msh : save as msh
        \item .crd .cnn .nbc(nodal BC) .srf
        \item cavity.def
        \item eightNodeBrick sixNodeWedge fourNodeTech
        \item simGmshCnvt -msh Case1.msh
\end{enumerate}
\subsection{Multiprocess}
    \begin{enumerate}
        \item InteractiveNode
        \item salloc --ntasks=16 --account=def-rjaiman --time=1:0:0 --mem-per-cpu=4G
        \item squeue
        \item multiple cpu task: simflow.config
        \item mpirun -n 16 ~/simflow-Nihar/bin/simflowOmpi -np 16
    \end{enumerate}

\section{13th May}
\begin{enumerate}
    \item Fix Case1 files -> the problem should be in the mesh file
    \item Check email for the lab access
    \item Run Case2: sucessfully run and the vortec shedding is observed
    \item Run Case3: VIV case with movable cylinder, follow the code project 1 article and try to reproduce the results.
    \item Run Case4: wave-run-up case
\end{enumerate}
\subsection{Steps to Simflow}
\begin{enumerate}
    \item Mesh - Geometry - 3D/2D w 1 layer thick - physical groups - nbc srf - Mesh.MshFileVersion = 2.13;
    \item Def file - geo -> msh - simflowCnvt 
    \item Post processing: \item scp conroyli@beluga.computecanada.ca:/home/conroyli/scratch/Case1-plate/debug1/*.vtk .
    \item Paraview :vtk
    \item MATLAB: Oisd Othd
    \item Restart Simflow: Rst
    \item 
\end{enumerate}
\subsection{Case4: Complie}
\begin{enumerate}
    \item src - solbc.c - change the height.. - save and "make" in src
    \item make clean to clear
    \item make again
    \item def - userdefined 
    \item 
\end{enumerate}
\subsection{Multiphase}
\begin{enumerate}
    \item Allen-Calm (Phase field)
    \item order parameter = $\phi$ from -1 to 1 (air to water)
    \item src-solpro.c 
\end{enumerate}

\section{14th May}
\begin{enumerate}
    \item /scratch/st-jelovica-1/ljc2018/
    \item ljc2018@sockeye.arc.ubc.ca cwl password
    \item 
\end{enumerate}
\subsection{Sockeye}
\begin{enumerate}
    \item make clean
    \item cp ../src-cc/src/y.* .
    \item make
\end{enumerate}
\subsection{Submit job}
\begin{enumerate}
    \item computecanadajob.sh
    \item sbatch
    \item make
\end{enumerate}
\section{15th May}
\begin{enumerate}
    \item Case 4 wave-run up submitted
    \item Case 3 VIV case submitted
\end{enumerate}

\section{16th May}
\begin{enumerate}
    \item Case 3 VIV case with corrected parameters submitted
\end{enumerate}

\section{21th May}
\begin{enumerate}
    \item N/A
\end{enumerate}
\section{22th May}
\subsection{Output}
\begin{enumerate}
    \item oisd --> Integrated Force
    \item othd --> Nodal Time History
    \item 
\end{enumerate}
\subsection{Initialization of multuphase}
\begin{enumerate}
    \item x is the longitutial, z is up and down
    \item solProc.c line  377 3i+2 is the z crd, need + or -1
    \item 
\end{enumerate}
\subsection{Killing waves}
\begin{enumerate}
    \item Method 1: ship at 0.25 L of the domain and l as the ship length, domain: 15-20l and 5l for the very coarse
    \item fine mesh in front, very coarse at the end
    \item Method 2: solTimeInteg
    \item damploc --> start location of damping
    \item 
\end{enumerate}
\subsection{multiple jobs}
\begin{enumerate}
    \item change simMakeInp --> run simMake --> make: complie c++ --> simflow and simflowOmpi
    \item make cleam + copy y files
    \item Wavetank: L=15+5m, H = 6m, W = 8m
    \item 
\end{enumerate}
Task: exaime waves from 0.2-2m 
Jobs: height 0.8m \& 1.6m and wave length 1m \& 2m
\begin{enumerate}
    \item 1: 0.8 \& 1
    \item 2: 0.8 \& 2
    \item 3: 1.6 \& 1
    \item 4: 1.6 \& 2
\end{enumerate}
\end{document}