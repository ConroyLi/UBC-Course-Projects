\documentclass[a4paper,11pt]{article} % The document class with options

\usepackage[margin=1in]{geometry}
\usepackage{newtxtext,newtxmath}
\usepackage[T1]{fontenc}
\usepackage{amsmath}
\usepackage{amsfonts}
\usepackage{microtype}
% chktex-file 3
% chktex-file 36

\begin{document}
\setlength{\parskip}{1em} 
\setlength{\parindent}{0pt}
\newcommand{\vect}[1]{\mathbf{#1}}

\title{MATH 521 Project Report \\
Review of Weak Galerkin Finite Element Method}
\author{Jincong Li \\ 60539939}
\date{Mar 28th}
\maketitle

Using arbitrary shapes of polygons or polyhedra for meshes in Finite Element Analysis (FEA) 
is particularly beneficial in situations requiring complex geometrical representations or 
when dealing with highly irregular domains. This flexibility allows for a more accurate 
approximation of curved or complex boundaries, improving the quality of the simulation 
without significantly increasing the computational cost. It's especially useful in adaptive 
mesh refinement processes where the mesh needs to evolve based on solution characteristics, 
allowing for efficient targeting of areas with high error or where finer resolution is 
needed to capture critical phenomena accurately.
\end{document}