\documentclass[12pt]{article} % The document class with options

\usepackage[margin=1in]{geometry}
\usepackage[utf8]{inputenc} 
\geometry{a4paper}
\usepackage{newtxtext,newtxmath}
\usepackage[T1]{fontenc}
\usepackage{amsmath}
\usepackage{amsfonts}
\usepackage{microtype}
\usepackage{graphicx}
\usepackage{listings} % For formatting and highlighting code
\usepackage{color}    % For colors in code highlighting

\begin{document}
\setlength{\parskip}{1em} 
\setlength{\parindent}{0pt}
\newcommand{\vect}[1]{\mathbf{#1}}

\begin{titlepage}  % This starts a title page environment
    \centering    % Center everything on the page

    %--- Add space at the top of the page ---
    \vspace*{2cm}
    
    %--- Title ---
    \normalsize \textbf{MEng Project Log} \\
    \vspace{0.5cm}  % Space between lines
    %\normalsize\textbf{Comparison of Gauss-Newton Method with Marquardt Modification with \\
    % Nelder-Mead Algorithm for Parameter Estimation in Algebraic Models} \\
    \vspace{2cm}  % Space between the title and the author name
    
    %--- Author ---
    \normalsize by\\
    \vspace{1cm}
    \normalsize Jincong Li \\ 
    \vspace{1cm}
    \normalsize M.Eng, The University of British Columbia, 2024
    \vspace{11cm}  % Space between the author and the date
    
    %--- Date ---
    \normalsize 8th May, 2024

    \vfill  % Push the following content to the bottom of the page
    %--- Bottom part of the page ---
    © Jincong Li, 2024
\end{titlepage}
\tableofcontents
\newpage
\section{8th May}
\begin{enumerate}
    \item Applied for the access to ICICS 227
    \item Applied for the access to compute canada
\end{enumerate}

\section{9th May}
\begin{enumerate}
    \item Installed Gmesh, Paraview and get access to compute canada
    \item Read Eigenfrequency analysis from COSMOL 
    \item Learn how to use Gmesh: geometry part and the mesh part with two case: vertical plane and cylinder
    \item Install the Simflow
\end{enumerate}
\section{10th May}
\subsection{Important Functions}
\begin{enumerate}
    \item simflow : run simflow to solve
    \item mpirun -nx simflowOmpi -npx
    \item simGmshCnvt -msh *.msh :mesh
    \item simPlt -type vtk : post processing
    \item gmsh -3 : convert geo to msh
    \item cd ~(home) .(current) ..(previous)
    \item vi :q :w :q! : quit, write, write\&quit, and quit without saving
    \item cp (source) (destination) :copy
    \item mv (source) (destination) :move
    \item scp :copy from others computer
    \item rm -r(folder)
    \item scp -r ineogi@beluga.computecanada.ca:~/scratch/CavityTutorial .
    \item scp conroyli@beluga.computecanada.ca:~/scratch/CavityTutorial/debug1/*.vtk .
\end{enumerate}
\subsection{Case 1: Lid Driven Cavity}
\begin{enumerate}
    \item 10 time steps with 0.1s
    \item saved in `debug1`'
    \item which simflow :give the location of the first version of simflow
    \item /simflow-Nihar/bin/simflow
    \item vi simflow.config
    \item simPlt -type vtk -min 0 -last 10 in CavityTutorial
\end{enumerate}
    \subsection{Files}
    \begin{enumerate} 
        \item cavity.geo
        \item cavity.msh : save as msh
        \item .crd .cnn .nbc(nodal BC) .srf
        \item cavity.def
        \item eightNodeBrick sixNodeWedge fourNodeTech
        \item simGmshCnvt -msh Case1.msh
        \item 
\end{enumerate}
\subsection{Files}
    \begin{enumerate}
        \item InteractiveNode
        \item salloc --ntasks=16 --account=def-rjaiman --time=1:0:0 --mem-per-cpu=4G
        \item squeue
        \item multiple cpu task: simflow.config
        \item mpirun -n 16 ~/simflow-Nihar/bin/simflowOmpi -np 16
    \end{enumerate}

\section{13th May}
\begin{enumerate}
    \item Fix Case1 files
    \item Check email for the lab access
\end{enumerate}

\end{document}