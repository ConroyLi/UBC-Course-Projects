\documentclass[17pt]{extarticle} % The document class with options

\usepackage[margin=1in]{geometry}
\usepackage{newtxtext,newtxmath}
%\usepackage[T1]{fontenc}
\usepackage{amsmath}
\usepackage{amsfonts}
\usepackage{microtype}
% chktex-file 3
% chktex-file 36

\begin{document}
\setlength{\parskip}{1em} 
\setlength{\parindent}{0pt}
S1
So my topic is about the weak Galerkin finite element method for second order elliptic
equations and for navier stokes equations.

S2
I will start from the background of FEM and then introduce the weak Galerkin FEM by explaining 
serveal key concepts such as the weak gradient and the overall algorithm. After that, just as what we have 
covered in the course material, we will look into the Existence and uniquess of the solution of weak
Galerkin FEM and then the general error analysis. And if time allowed, we shall see how the WG can be 
applied for NS equations.

\section*{Background}
S4
Typical FEM can be classified into two categories. The first category focus on approximating the 
solution u of the PDE directly, such as the Standard Galerkin FEM which 
use variational principles to approximate the solution u by minimizing an energy functional.
It typically requires the solution space to be a subset of the Sobolev space H 1.

Another example is the Interior Penalty Type Discontinuous Galerkin Methods: These are a class of Galerkin methods that allow 
for discontinuities in the solution across the boundaries of the elements in the discretized domain. 
They are particularly useful for dealing with high-contrast media or when higher-order polynomials 
are used for the approximation. The method involves adding penalty terms to the formulation to 
enforce continuity constraints weakly.

The second category involve formulating the PDE problem by introducing auxiliary variables, 
typically representing physical quantities like flux, and then seeking solutions for
both the primary variable u and the auxiliary variable(s). This approach can lead to 
systems of equations that capture more physical properties directly, such as conservation 
laws.

One example is the Standard Mixed Finite Elements. This approach 
is beneficial for problems where maintaining local conservation laws is important. 

Another example is the Various Discontinuous Galerkin Methods Based on Both Variables: These methods extend 
the discontinuous Galerkin framework to handle both the primary variable and auxiliary 
variables. Notice that this approach combines the advantages of discontinuous Galerkin 
methods (such as flexibility in handling complex geometries and material discontinuities) 
with the ability to approximate additional physical quantities directly. However, 
the author of the paper Iam refering to stated taht the WG method is different from these methods when the coefficients are general variable functions.

The weak Galerkin FEM is closely related to the mixed finite element method. Actually, Under some conditions, 
they are equaivalent, and also the WG is close to the hybridizable discontinuous
Galerkin (HDG) method when the relavant coefficients are set to zero or constant. 

When I was firstly brosing over all the articals related to WG, I did not notice this comparison, but after I selected
paper and went through them carefully, I find it is weird of the author the reference paper to explicitly put a sentence in
the introduction section stating that WG is different from HDG. So I did a boarder search then I found something that are 
interesting. 

S5
So as we can see here, the very first paper from Dr Wang on weak galerkin is published in 2013, and he succecively published 
plenty of paper related to WG. But in 2018, another Dr B seems like not really agree with those articles and published a paper
which bascially stating that the WG is equaivalent to HDG. And then not surprisingly, Dr Wang notices this and published another
paper on the same website in the next year to declare his thoughts. So I browsed these two articles and tring to figure it out, however, unfortunately,
though those two paper are not long both around 20 pages, I found they are too technical for me interpolate them correctly since my
background knowledge of analysis is really poor. But from the viewpoint of an engineering student, I think as long as the method 
works as intened and well, it is enough and the method is considered to be a good method. But, I know this might be of interest
for some mathmatians, so I left is here as part of my findings.


\section*{Intro}

S7
the differential operators that are used to define the variational or weak form are locally reconstructed on each element using
problem-independent tools. This could form a set of building blocks which may
constitute a WG calculus;

S8
In the standard Galerkin method, the trial space and the test space are each replaced by properly 
defined subspaces of finite dimensions. The resulting solution in the subspace/subset is
called a Galerkin approximation. A key feature in the Galerkin method is that the approximating 
functions are chosen in a way that the gradient operator can be successfully applied to them 
in the classical sense. For instance,the approximating functions are continuous piecewise polynomials 
over a prescribed finite element partition for the domain. 

S9
But in some cases, we may not have this property. Thus, we need weak Galerkin FEM. By stating this, one might still
wondering why or when we need WG.  
Here are some potiential cases that WG might be useful, notice that the reason why WG could be useful is not only 
the fact that it only requires weak regularties but also its stability compared to the traditional FEM such as in the advection
situation. Howeverm, if other FEM are doing their job perfectly in some fields then donot bother even considering WG FEM

S10
WG which uses a weak gradient operator. To introduce weak gradient, first we need to know what is weak function, 
let K be any polygonal domain with interior K0 and boundary partial K. A weak function on the
region K refers to a function defined like this, The first component v0 can be understood
as the value of v in the interior of K, and the second component vb is the value of v on the boundary of K. Note that vb may
not be necessarily related to the trace of v0. And then the corresponding weak function space looks
like this.

Thus, for any weak function v belongs the space W(K), the weak gradient of v can be defined as this.
So now, With the help of this weak gradient operator, the trial and test functions can be allowed to take 
separate values/definitions on the interior of each element T and its boundary.

S11
if the components of v are restrictions of a function u belongs of H1 K on K0 and partial K, respectively, then we would have this steps and notice here the nabla v is equivalent
to nabla u which is the classical gradient of u.

With the weak gradient operator as introduced in this section, derivatives can be taken for functions without any continuity across the
boundary of each triangle. Thus, the concept of weak gradient allows the use of generalized functions in approximations.

S12
Now we need the discretized form of the gradient opeartor. We first define two things, the first one is 
the set of polynomials of the domain K with degree no more than r and then the subspace of those polynomials.
So the discrete weak gradient operator with respect to degree r is defined as this equation. 

S13
After we introduced all key concepts for weak Galerkin FEM, one can summerize it into the following setups,
but with two major difference from the standard Galerkin FEM we learned, first...
second....

S14
So now we define two sets of polynomials for the interior of each T and on the boundary of each T.
And then the discrete weak function space could be defined as such, and its corresponding weak finite element
space with vanishing boundary conditions could be stated as this. 

S15
Therefore, in general, the bilinear form is formulated as this. 

S16
With all of these concepts defined, we could summerize this Weak Galerkin FEM in to such statement:

S17
OK, now it is the time to discuss the existence and uniqueness of the WG, 

\end{document}