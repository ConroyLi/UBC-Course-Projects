\documentclass[a4paper,12pt]{article} % The document class with options

\usepackage[T1]{fontenc}
\usepackage{amsmath}
\usepackage{amsfonts}
\usepackage{microtype}
% chktex-file 3
% chktex-file 8
% chktex-file 13
% chktex-file 18
\begin{document}
\setlength{\parskip}{1em} 
\setlength{\parindent}{0pt}
\section{Main Idea}
2D, wave generator, FSI, homogenous or non-homogenous ship/boat, free surface between fluid and air, gravity



\section{Experimental and numerical study of the vertical motions
of a bulk carrier and a Ro–Ro ship in extreme waves s40722-015-0019-1}
Some reports from B C. Guedes Soares
c.guedes.soares@centec.tecnico.ulisboa.pt
1 Centre for Marine Technology and Ocean Engineering
(CENTEC), Instituto Superior Técnico, Universidade de
Lisboa, Av. Rovisco País, 1049-001 Lisbon, Portugal
cruise vessels encountering rogue waves, such as Grand
Voyager (February 2005) (Bertotti and Cavaleri 2008), Norwegian
Dawn (April 2005) (Didenkulova et al. 2006) and MS
Louis Majesty (March 2010) (Cavaleri et al. 2012), confirm that ships might be exposed to extreme environment conditions
during its lifetime, including encounters with abnormal
waves, whose damaging effects have already been pointed
out by Faulkner and Buckley (1997).

The wave generator is fully computer controlled
and a software is implemented which enables the generation
of transient wave packages, deterministic irregular sea
states with predefined characteristics as well as tailor-made
critical wave sequences (Clauss and Kühnlein 1996; Clauss
and Schmittner 2005). The technique has been established to
reproduce a large variety of wave sequences such as single
abnormal waves as well as groups of rogue waves embedded
in irregular sea states to investigate the response of floating
structures to an extreme, but realistic, wave environment
(Clauss et al. 2004).

I could use the model of wave in this paper to simulate the real situation. 


\section{Experiments and computations of solitary wave interaction with fixed,
partially submerged, vertical cylinders s40722-019-00137-8}
wave presented but finite difference ,3D, rigid vertical cylinder submerged in water

\section{Experiments and calculations of cnoidal wave loads on a flat plate
in shallow-water /s40722-014-0007-x}
vertial flat plate, more experimental

\section{Simulation of floating structure dynamics in waves by implicit
coupling of a fully non-linear potential flow model and a rigid
body motion approach s40722-014-0006-y}
2D potential flow floating

The floating idea comes here
\section{PREDICTION OF GLOBAL WHIPPING RESPONSES ON A LARGE CRUISE SHIP
UNDER UNKNOWN SEA STATES USING AN LSTM BASED ENCODER DECODER MODELOMAE
202 4 126186}
The document presents a comprehensive 
study on predicting global whipping responses on a large cruise ship under 
unknown sea states using an LSTM $($Long Short-Term Memory$)$based encoder-decoder
model. It highlights the significance of accurately predicting vertical bending 
moments for ship safety and introduces a novel approach employing a machine learning 
model trained on a dataset including motion data and vertical bending moment history 
from numerical simulations. The study verifies the model's effectiveness under both 
known and unknown sea states and explores a data mixing strategy to enhance prediction 
accuracy for unknown conditions. This approach demonstrates potential in improving ship 
safety by enabling accurate predictions of critical structural responses under varying 
sea conditions.

\section{Viscous fluid–flexible structure interaction analysis on ship
springing and whipping responses in regular waves}
The document titled "Viscous fluid–flexible structure interaction analysis 
on ship springing and whipping responses in regular waves" explores the dynamics
of ship responses to wave-induced forces through a comprehensive computational 
fluid dynamics (CFD) and finite element analysis (FEA) approach. This study employs
a two-way fluid-structure interaction (FSI) methodology to predict ship motions,
wave loads, and hydroelastic responses in regular waves, focusing on the springing 
and whipping phenomena that are critical for ship structural integrity. The research
presents an in-depth analysis of the ship's global motions, vertical accelerations,
and wave-induced vertical bending moments, highlighting the capability of the CFD-FEA
co-simulation method to replicate complex flow phenomena and ship responses accurately.
\section{}
\section{}
\section{}
\section{}
\section{}


\end{document}