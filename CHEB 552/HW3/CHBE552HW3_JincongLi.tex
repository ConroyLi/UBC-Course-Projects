\documentclass[a4paper,12pt]{article} % The document class with options

\usepackage[margin=1in]{geometry}
\usepackage{newtxtext,newtxmath}
\usepackage[T1]{fontenc}
\usepackage{amsmath}
\usepackage{amsfonts}
\usepackage{microtype}
% chktex-file 3
% chktex-file 36

\begin{document}
\setlength{\parskip}{1em} 
\setlength{\parindent}{0pt}
\newcommand{\vect}[1]{\mathbf{#1}}

\title{CHBE 552 Problem Set 1}
\author{Jincong Li \\ 60539939}
\date{Feb 24th}
\maketitle

\section*{\textbf{Question 1}}
By implementing SA method with following parameters: bounds $= (-1, 1)$, initial temperature $= 10000$, cooling rate $= 0.95$, stopping temperature $= 0.001$, and max iterations $= 1000$,
three tests are conducted and the results are shown as following:

Test 1:

Best $x_1$ and $x_2$ are $[ 0.00016685, -0.00014399]$ and the
best function value is $1.074285020 \cdot 10^{-6}$.

Test 2:

Best $x_1$ and $x_2$ are $[0.00019128, 0.00056343]$ and the
best function value is $1.08674897771 \cdot 10^{-5}$.

Test 3:

Best $x_1$ and $x_2$ are $[0.00023714, 0.00051509]$ and the
best function value is $9.45038628186 \cdot 10^{-6}$.

Thus, one can conclude the best $x_1$ and $x_2$ are $[0, 0]$ and the
best function value is $0$. The tiny error might comes from the numerical uncertainty of float computation in Python.

\section*{\textbf{Question 2}}

By implementing Luus-Jaakola method accoording to the reference with such parameters:
\begin{align*}
    \text{Initial guess of each $x_j$:} x_{bounds} = [(0, 1)]\\
    \text{max iterations:} N=10000\\
    \text{Initial search area:} V_{initial}=0.1\\
    \text{Search area reduction factor:} V_{r}=0.999
\end{align*}
Solution for variables $x_1$ to $x_{10} $ are found to be 
\begin{align*}
    x_1 &= 0.0406700206 \\
    x_2 &= 0.147744627 \\
    x_3 &= 0.783102950 \\
    x_4 &= 0.00141387372 \\
    x_5 &= 0.485248112 \\
    x_6 &= 0.000691530012 \\
    x_7 &= 0.0273983722 \\
    x_8 &= 0.0179474168 \\
    x_9 &= 0.0373039831 \\
    x_{10} &= 0.0969432948
\end{align*}
And the optimal objective function value is computed to be -47.7610, which generally agrees with the value provided
in the reference paper.

Note that the number of iterations are setted to be very such that a global minimum could be found, and the search area
reduction fractor are close to 1 such that the search area only reduces a little bit in each iteration, thus, the solution of 
$x_j$ could have more digits.

\section*{\textbf{Question 3}}

By implementing Luus-Jaakola method accoording to the reference with same parameters stated in question 2, except for initial guess of 
$x_j$ since the bounds should follow the indicated constraint in the paper:
\begin{align*}
\text{Best Solution for independent variable:} \\
x_1 &= 1727.77228\\
x_7 &= 94.25960\\
x_8 &= 10.41761\\
\text{Dependent Variables:}\\
x_2 &= 0.000160000000 \\
x_3 &=  0.994398201\\
x_4 &=  0.00305494307\\
x_5 &=  0.00199925826\\
x_6 &= 0.908309867 \\
x_9 &= 2.56910919 \\
x_{10} &= 0.0149778787 \\
\text{Best Objective Function Value:} 1161.45677
\end{align*}
The results generally agrees with the value provided in the reference paper.
\section*{\textbf{Question 4\&5}}
By implementing Nelder-Mead algorithm in Python, for function in part 1: the minimum point is $[0.99999871\approx 1, 0.99999707\approx 1]$, and the
minimum function value is $1.3556651538450528e-11 \approx 0 $.

For function in part 2: the minimum point is $[ 6.50737582e-04\approx 0, -6.52767435e-05\approx 0, -6.74455549e-04\approx 0, -6.76241743e-04\approx 0]$, and the
minimum function value is $5.379451007433241e-11 \approx 0 $.

Again, The tiny error might comes from the numerical uncertainty of float computation in Python.

\section*{\textbf{Question 6}}

% Step 1: Express the work done by the first compressor C-1
The work done by the first compressor \( C-1 \) is given by:
\begin{equation}
W_{C-1} = \frac{RT_1}{\gamma} \left( \left( \frac{P_2}{P_1} \right)^\gamma - 1 \right)
\end{equation}

% Step 2: Express the work done by the second compressor C-2
The work done by the second compressor \( C-2 \), compressing from \( P_2 \) to \( P_4 \), is (since$P_2 = P_3$):
\begin{equation}
W_{C-2} = \frac{RT_1}{\gamma} \left( \left( \frac{P_4}{P_2} \right)^\gamma - 1 \right)
\end{equation}

% Step 3: Find the total work W_tot
The total work \( W_{tot} \) is the sum of the work done by both compressors:
\begin{equation}
W_{tot} = W_{C-1} + W_{C-2}
\end{equation}
\begin{equation}
W_{tot} = \frac{RT_1}{\gamma} \left( \left( \frac{P_2}{P_1} \right)^\gamma - 1 \right) + \frac{RT_1}{\gamma} \left( \left( \frac{P_4}{P_2} \right)^\gamma - 1 \right)
\end{equation}

% Step 4: Differentiate W_tot with respect to P_2
Differentiate \( W_{tot} \) with respect to \( P_2 \) to find the minimum:
\begin{equation}
\frac{dW_{tot}}{dP_2} = \frac{RT_1 \gamma}{\gamma} \left( \frac{P_2}{P_1} \right)^{\gamma-1} \frac{1}{P_1} - \frac{RT_1 \gamma}{\gamma} \left( \frac{P_4}{P_2} \right)^{\gamma} \frac{1}{P_2^2}
\end{equation}

% Step 5: Solve for optimal P_2
Set the derivative equal to zero and solve for \( P_2 \):
\begin{equation}
0 = \frac{RT_1 \gamma}{\gamma} \left[ \left( \frac{P_2}{P_1} \right)^{\gamma-1} \frac{1}{P_1} - \left( \frac{P_4}{P_2} \right)^{\gamma} \frac{1}{P_2^2} \right]
\end{equation}

Simplify and solve for \( P_2 \):
\begin{align*}
0 &= P_2^{\gamma-1} \frac{1}{P_1^\gamma} - P_4^\gamma \frac{1}{P_2^{\gamma+1}} \\
P_4^\gamma &= P_2^{2\gamma} \frac{1}{P_1^\gamma} \\
P_2^{2\gamma} &= P_1^\gamma P_4^\gamma \\
P_2^2 &= P_1 P_4
\end{align*}

Take the square root of both sides to find \( P_2 \):
\begin{equation}
P_2 = \sqrt{P_1 P_4}
\end{equation}

This result indicates that the optimal intermediate pressure \( P_2 \) is the geometric mean of the inlet pressure \( P_1 \) and the final pressure \( P_4 \).



\end{document}