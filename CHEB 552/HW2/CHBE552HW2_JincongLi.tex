\documentclass[a4paper,12pt]{article} % The document class with options

\usepackage[T1]{fontenc}
\usepackage{amsmath}
\usepackage{amsfonts}
\usepackage{microtype}
% chktex-file 3
% chktex-file 36

\begin{document}
\setlength{\parskip}{1em} 
\setlength{\parindent}{0pt}
\newcommand{\vect}[1]{\mathbf{#1}}

\title{CHBE 552 Problem Set 1}
\author{Jincong Li \\ 60539939}
\date{Feb 24th}
\maketitle

\section*{\textbf{Question 1}}

\subsection*{Part a}
\begin{align*}
    \vect{F_1} &= x_{1}^{2} + x_{1} x_{2} + x_{1} x_{3} + x_{1} x_{4} + x_{1}\\
    & + x_{2}^{2} + x_{2} x_{3} + x_{2} x_{4} + x_{2} + x_{3}^{2} + x_{3} x_{4}\\
    & + x_{3} + x_{4}^{2} + x_{4} + 1 \\
    \vect{H_{F_1}} &= \left[\begin{matrix}2 & 1 & 1 & 1\\1 & 2 & 1 & 1\\1 & 1 & 2 & 1\\1 & 1 & 1 & 2\end{matrix}\right] \\
    \vect{F_2} &= 8 x_{1}^{2} + 4 x_{1} x_{2} + 5 x_{2}^{2} \\
    \vect{H_{F_2}} &= \left[\begin{matrix}16 & 4\\4 & 10\end{matrix}\right] \\
    \intertext{By implementating Newton's method, the corresponding minimums are found to be:}
    \vect{x_{F1}} &= \left[\begin{matrix}-0.2\\-0.2\\-0.2\\-0.2\end{matrix}\right] \text{or} \  \left[\begin{matrix}-0.2\\-0.2\\-0.2\\-0.2\end{matrix}\right]\\
    \vect{x_{F2}} &= \left[\begin{matrix}0\\0\end{matrix}\right]
    \intertext{Note that for $\vect{F_1}$, two initial guesses are provided the value of $\vect{x_{F1}}$ shown above
    are derived from thoes two initial guesses and they are actually the same.}
\end{align*}

\subsection*{Part b}
\begin{align*}
    \vect{F} &= 4 \left(x_{1} - 5\right)^{2} + \left(x_{2} - 6\right)^{2} \\
    \nabla \vect{F}&= \left[ 8 x_{1} - 40, \  2 x_{2} - 12\right]\\
    \intertext{By implementating Fletcher Reeves's method, the minimum is found to be:}
    \vect{x} &= \left[\begin{matrix}5.11\\2.48\end{matrix}\right]
\end{align*}

\subsection*{Part c}
\begin{align*}
    \vect{F} &= 2 x_{1}^{2} + 2 x_{1} x_{2} + x_{1} + x_{2}^{2} - x_{2} \\
    \nabla \vect{F}&= \left[ 4 x_{1} + 2 x_{2} + 1, \  2 x_{1} + 2 x_{2} - 1\right]\\
    \intertext{By implementating DFP method, the minimum is found to be:}
    \vect{x} &= \left[\begin{matrix}-0.65\\0.98\end{matrix}\right]
\end{align*}
\newpage
\section*{\textbf{Question 2}}

\subsection*{Part a}
\begin{align*}
    F &= x_{1}^{2} + 10 x_{1} + x_{2}^{2} + 20 x_{2} + 25 \\
    L &= \lambda \left(x_{1} + x_{2}\right) + x_{1}^{2} + 10 x_{1} + x_{2}^{2} + 20 x_{2} + 25 \\
    \partial F_{x_1} &= \lambda + 2 x_{1} + 10 \\
    \partial F_{x_2} &= \lambda + 2 x_{2} + 20 \\
    \partial F_{\lambda} &= x_{1} + x_{2} \\
    \intertext{By setting those partial derivative equations to zero, the optimum values are found to be:}
    x_{1,opt} &= \frac{5}{2} \\
    x_{2,opt} &= - \frac{5}{2} \\
    \lambda_{opt} &= -15 \\
    \intertext{At this optimum point the original function is evaluated to be:}
    F_{opt} &= \frac{25}{2} \\
    \intertext{For a sensitivity test the constraint condition is changed to $x_1 + x_2 = 0.01$, then the previous
    steps are repeated:}
    L_{new} &= \lambda \left(x_{1} + x_{2} - 0.01\right) + x_{1}^{2} + 10 x_{1} + x_{2}^{2} + 20 x_{2} + 25 \\
    \partial F_{x_1} &= \lambda + 2 x_{1} + 10 \\
    \partial F_{x_2} &= \lambda + 2 x_{2} + 20 \\
    \partial F_{\lambda} &= x_{1} + x_{2} - 0.01 \\
    x_{1,opt} &= 2.505 \\
    x_{2,opt} &= -2.495 \\
    \lambda_{opt} &= -15.01\\
    F_{new,opt} &= 12.65005 \\
    \intertext{Thus, the increment of the function value is computed to be}
    \Delta F &= 0.15
\end{align*}

\subsection*{Part b}
\begin{align*}
    F &= - \pi x_{1}^{2} x_{2} \\
    L &= \lambda \left(2 \pi x_{1}^{2} + 2 \pi x_{1} x_{2} - 24 \pi\right) - \pi x_{1}^{2} x_{2} \\
    \partial F_{x_1} &= \lambda \left(4 \pi x_{1} + 2 \pi x_{2}\right) - 2 \pi x_{1} x_{2} \\
    \partial F_{x_2} &= 2 \pi \lambda x_{1} - \pi x_{1}^{2} \\
    \partial F_{\lambda} &= 2 \pi x_{1}^{2} + 2 \pi x_{1} x_{2} - 24 \pi \\
    \text{Solution} &= \left[ \left( 2, \  4\right)\right]
\end{align*}
\newpage
\section*{\textbf{Question 3}}
As implemted in Python, the optimal x and minimum value of objective function are shown below:

Optimal x in order from $x_1$ to $x_{10}$: 

0.0069648 \\
0.068068\\
0.90717\\
0.0003625\\
0.49078\\
0.00048055\\
0.017593\\
0.0029156\\
0.01512\\
0.042083

Minimum f(x): -43.495
\newpage
\section*{\textbf{Question 4}}
\begin{align*}
    F &= y^{2} + \left(x - 2\right)^{2} + \left(z - 1\right)^{2} \\
    L &= \lambda \left(- 4 x^{2} - 2 y^{2} + z^{2}\right) + y^{2} + \left(x - 2\right)^{2} + \left(z - 1\right)^{2} \\
    \partial F_{x} &= - 8 \lambda x + 2 x - 4 \\
    \partial F_{y} &= - 4 \lambda y + 2 y \\
    \partial F_{z} &= 2 \lambda z + 2 z - 2 \\
    \partial F_{\lambda} &= - 4 x^{2} - 2 y^{2} + z^{2} \\
    \vect{x} &=   \left( \frac{4}{5}, \  0, \  \frac{8}{5}\right) \\
    \intertext{With $\lambda = - \frac{3}{8}$}
\end{align*}
\newpage
\section*{\textbf{Question 5}}
\subsection*{Part a}
\begin{align*}
    F &= x_{1}^{2} - 14 x_{1} + x_{2}^{2} - 6 x_{2} - 7 \\
    L &= \lambda_{1} \left(x_{1} + x_{2} - 2\right) + \lambda_{2} \left(x_{1} + 2 x_{2} - 3\right) \\
      &+ x_{1}^{2} - 14 x_{1} + x_{2}^{2} - 6 x_{2} - 7 \\
    \partial F_{x} &= \lambda_{1} + \lambda_{2} + 2 x_{1} - 14 \\
    \partial F_{y} &= \lambda_{1} + 2 \lambda_{2} + 2 x_{2} - 6 \\
    \partial F_{\lambda_1} &= x_{1} + x_{2} - 2 \\
    \partial F_{\lambda_2} &= x_{1} + 2 x_{2} - 3 \\
    \intertext{Determined by KTC, the only optimal point can be found to be:}
    \vect{x} &= \left\{ x_{1} : 3, \  x_{2} : -1\right\} \\
    \intertext{With}
    \lambda_{1,2} &= \left\{ \lambda_{1} : 8, \  \lambda_{2} : 0\right\}
    \intertext{Note here the Lagrangian Multiplier for the second inequality constraint is determined to be zero,
    which indicates that it is inactive constraint. And the minimum objective function value is computed to be $-33$.}
    \intertext{For the sufficient condition, the Hessian matrix evaluated at the optimal point is determined to be:}
    H^* & = \left[\begin{matrix}2 & 0\\0 & 2\end{matrix}\right]\\
    \intertext{The corresponding eigen values are:}
    \lambda_{1,2} &=\left\{ 2 , 2\right\}
    \intertext{Since both eigen values are positve, the objective function is considered convex at the optimal point, thus,
    the optimal point is indeed a true global minimum.}
\end{align*}

\subsection*{Part b}
\begin{align*}
    F &= x_{1}^{2} + 2 \left(x_{2} + 1\right)^{2} \\
    L &= \lambda_{1} \left(- x_{1} + x_{2} - 2\right) + \lambda_{2} \left(- x_{1} - x_{2} - 1\right) + x_{1}^{2} + 2 \left(x_{2} + 1\right)^{2} \\
    \partial F_{x} &= - \lambda_{1} - \lambda_{2} + 2 x_{1} \\
    \partial F_{y} &= \lambda_{1} - \lambda_{2} + 4 x_{2} + 4\\
    \partial F_{\lambda_1} &= - x_{1} + x_{2} - 2\\
    \partial F_{\lambda_2} &= - x_{1} - x_{2} - 1\\
    \vect{x} &= \left\{x_{1} : - \frac{3}{2}, \  x_{2} : \frac{1}{2}\right\} \\
    \intertext{With}
    \lambda_{1,2} &= \left\{ \lambda_{1} : - \frac{9}{2}, \  \lambda_{2} : \frac{3}{2}\right\}
    \intertext{And the minimum objective function value is computed to be $\frac{27}{4}$.}
    \intertext{For the sufficient condition, the Hessian matrix evaluated at the optimal point is determined to be:}
    H^* & = \left[\begin{matrix}2 & 0\\0 & 4\end{matrix}\right]\\
    \intertext{The corresponding eigen values are:}
    \lambda_{1,2} &=\left\{ 2 , 4 \right\}
    \intertext{Since both eigen values are positve, the objective function is considered convex at the optimal point, thus,
    the optimal point is indeed a true global minimum.}
\end{align*}
\newpage
\section*{\textbf{Question 6}}
\begin{align*}
    F &= x_{1}^{2} + x_{1} x_{2} - 4 x_{1} + 1.5 x_{2}^{2} - 7 x_{2} - \log{\left(x_{1} \right)} - \log{\left(x_{2} \right)} + 9 \\
    L_a &= x_{1}^{2} + x_{1} x_{2} - 4 x_{1} + 1.5 x_{2}^{2} - 7 x_{2} - \log{\left(x_{1} \right)} - \log{\left(x_{2} \right)} + 9\\
    L_b &= \lambda_{1} \left(- x_{1} x_{2} + 4\right) + x_{1}^{2} + x_{1} x_{2} - 4 x_{1} + 1.5 x_{2}^{2} - 7 x_{2} - \log{\left(x_{1} \right)} - \log{\left(x_{2} \right)} + 9\\
    L_c &= \lambda_{1} \left(- x_{1} x_{2} + 4\right) + \lambda_{2} \cdot \left(2 x_{1} - x_{2}\right) + x_{1}^{2} + x_{1} x_{2} - 4 x_{1} + 1.5 x_{2}^{2} - 7 x_{2} - \log{\left(x_{1} \right)} - \log{\left(x_{2} \right)} + 9\\
    \partial L_{a,x_1} &= 2 x_{1} + x_{2} - 4 - \frac{1}{x_{1}}\\
    \partial L_{a,x_2} &= x_{1} + 3.0 x_{2} - 7 - \frac{1}{x_{2}}\\
    \partial L_{b,x_1} &= - \lambda_{1} x_{2} + 2 x_{1} + x_{2} - 4 - \frac{1}{x_{1}}\\
    \partial L_{b,x_2} &= - \lambda_{1} x_{1} + x_{1} + 3.0 x_{2} - 7 - \frac{1}{x_{2}}\\
    \partial L_{b,\lambda_1} &= - x_{1} x_{2} + 4\\
    \partial L_{c,x_1} &= - \lambda_{1} x_{2} + 2 \lambda_{2} + 2 x_{1} + x_{2} - 4 - \frac{1}{x_{1}}\\
    \partial L_{c,x_2} &= - \lambda_{1} x_{1} - \lambda_{2} + x_{1} + 3.0 x_{2} - 7 - \frac{1}{x_{2}}\\
    \partial L_{c,\lambda_1} &= - x_{1} x_{2} + 4\\
    \partial L_{c,\lambda_2} &= 2 x_{1} - x_{2}\\
    \intertext{Note that since the partial derivative equations are not linear anymore, the following results are 
    computed numerically.
    For part a:}
    \vect{x_a} &= \left[\begin{matrix}1.34754858228762\\2.04699110826639\end{matrix}\right] \\
    \intertext{For part b:}
    \vect{x_b} &= \left[\begin{matrix}1.79811994048488\\2.22454571018291\end{matrix}\right] \\
    \intertext{With}
    \lambda_{1} &= \left[\begin{matrix}0.568497719699798\end{matrix}\right] \\
    \intertext{For part c:}
    \vect{x_c} &= \left[\begin{matrix}1.4142135623731\\2.82842712474619\end{matrix}\right] \\
    \intertext{With}
    \lambda_{1,2} &= \left[\begin{matrix}1.06801948466054\\1.03553390593274\end{matrix}\right]
    \intertext{And the minimum objective function value is computed to be:}
    F_a &= -0.874224083186354\\
    F_b &= -0.494453348930655\\
    F_c &=  0.157861516164399
    \intertext{For the sufficient condition, the Hessian matrixs for each condition evaluated at
     the optimal point are determined to be:}
    H_a^* &= \left[\begin{matrix}2.55069500468907 & 1\\1 & 3.23865365370371\end{matrix}\right]\\
    H_b^* &= \left[\begin{matrix}2.30928772604332 & 1\\1 & 3.20207720752309\end{matrix}\right]\\
    H_c^* &= \left[\begin{matrix}2.5 & 1\\1 & 3.125\end{matrix}\right]\\
    \intertext{The corresponding eigen values are:}
    \lambda_{a,1,2} &=\left\{ 1.8371669884708  ,  3.95218166992198 \right\} \\
    \lambda_{b,1,2} &=\left\{ 1.66057139270888 ,  3.85079354085753 \right\} \\
    \lambda_{c,1,2} &=\left\{ 1.76480908660999 ,  3.86019091339001 \right\}
    \intertext{Since all eigen values for each case are positve, the objective function is considered 
    convex at the optimal points, thus, the optimal points are indeed true global minimum.}
\end{align*}

\end{document}