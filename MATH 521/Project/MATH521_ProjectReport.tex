\documentclass[a4paper,11pt]{article} % The document class with options

\usepackage[margin=1in]{geometry}
\usepackage{newtxtext,newtxmath}
\usepackage[T1]{fontenc}
\usepackage{amsmath}
\usepackage{amsfonts}
\usepackage{microtype}
% chktex-file 3
% chktex-file 36

\begin{document}
\setlength{\parskip}{1em} 
\setlength{\parindent}{0pt}
\newcommand{\vect}[1]{\mathbf{#1}}

\title{MATH 521 Project Report \\
Review of Weak Galerkin Finite Element Method}
\author{Jincong Li \\ 60539939}
\date{Mar 28th}
\maketitle

Using arbitrary shapes of polygons or polyhedra for meshes in Finite Element Analysis (FEA) 
is particularly beneficial in situations requiring complex geometrical representations or 
when dealing with highly irregular domains. This flexibility allows for a more accurate 
approximation of curved or complex boundaries, improving the quality of the simulation 
without significantly increasing the computational cost. It's especially useful in adaptive 
mesh refinement processes where the mesh needs to evolve based on solution characteristics, 
allowing for efficient targeting of areas with high error or where finer resolution is 
needed to capture critical phenomena accurately.
\section*{Abstract}

\section*{Introduction}
This category includes numerical methods that focus on approximating the solution u of the PDE directly. 
These methods construct a numerical scheme by working primarily with the variable 
u, which represents the quantity of interest in the PDE. Examples include:
Standard Galerkin Finite Element Methods: These methods use variational principles to approximate the solution 
u by minimizing an energy functional. They typically require the solution space to be a subset of the Sobolev space 
1 H 1
, implying that the solution is sought in a space of functions that are square-integrable along with their 
first derivatives. References [1–3] in the document likely refer to foundational texts and research papers 
that establish the theory and application of Galerkin methods in solving elliptic PDEs.

Interior Penalty Type Discontinuous Galerkin Methods: These are a class of Galerkin methods that allow 
for discontinuities in the solution across the boundaries of the elements in the discretized domain. 
They are particularly useful for dealing with high-contrast media or when higher-order polynomials 
are used for the approximation. The method involves adding penalty terms to the formulation to 
enforce continuity constraints weakly. References [4–8] would detail the development and analysis of these methods.

These methods involve formulating the PDE problem by introducing auxiliary variables, 
typically representing physical quantities like flux, and then seeking solutions for
both the primary variable u and the auxiliary variable(s). This approach can lead to 
systems of equations that capture more physical properties directly, such as conservation 
laws. Examples include:

Standard Mixed Finite Elements: In mixed finite element methods, the solution involves 
not only the primary variable u but also other variables such as the flux. This approach 
is beneficial for problems where maintaining local conservation laws is important. 
The method allows for the direct approximation of quantities that would otherwise 
be derived quantities in standard methods. References [9–16] in the document likely 
cover seminal and contemporary research on mixed methods.

Various Discontinuous Galerkin Methods Based on Both Variables: These methods extend 
the discontinuous Galerkin framework to handle both the primary variable and auxiliary 
variables like flux. This approach combines the advantages of discontinuous Galerkin 
methods (such as flexibility in handling complex geometries and material discontinuities) 
with the ability to approximate additional physical quantities directly. 
References [17–20] would explore variations and applications of these methods.

In the standard Galerkin method, the trial space H1($\Omega$) and the test 
space $H^1_0$ ($\Omega$) in (1.4) are each replaced by properly defined
subspaces of finite dimensions.

\section*{Preliminaries and notations}

\section*{Weak Gradient Operator}

\section*{A weak Galerkin finite element method}

\section*{Existence and uniqueness for WG approximations}

\section*{Error analysis}

\end{document}