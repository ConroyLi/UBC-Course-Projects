\documentclass[a4paper,12pt]{article} % The document class with options

\usepackage[T1]{fontenc}
\usepackage{amsmath}
\usepackage{amsfonts}
\usepackage{microtype}
% chktex-file 3
% chktex-file 13
\begin{document}
\setlength{\parskip}{1em} 
\setlength{\parindent}{0pt}
\newcommand{\vect}[1]{\mathbf{#1}}

\title{MECH 570C Assigment 2}
\author{Jincong Li 60539939}
\date{Feb.6th}
\maketitle
\section{\textbf{Question 1}}

% In the context of continuum mechanics, when deformations and displacements are infinitesimally small, the distinction between Lagrangian (material) and Eulerian (spatial) descriptions becomes negligible. This simplification allows for linear approximations of key quantities like the deformation gradient \(F\), its determinant \(J\), and the strain tensor \(E\).

\subsection{Deformation Gradient (\(F\))}

% The deformation gradient tensor \(F\) relates material points in the undeformed configuration to the deformed configuration and is defined as:
\[ F_{ij} = \frac{\partial x_i}{\partial X_j} \]

Note \(x_i\) are the spatial coordinates in the deformed configuration and \(X_j\) are the material coordinates in the undeformed (reference) configuration.

% For infinitesimally small deformations:
The displacement vector \(u_i\) is defined as following:
\[ u_i = x_i - X_i \]
So the deformation vector becomes:
\[ F_{ij} = \frac{\partial (X_i + u_i)}{\partial X_j} = \delta_{ij} + \frac{\partial u_i}{\partial X_j} \]

% Here, \(\delta_{ij}\) is the Kronecker delta, which is 1 if \(i=j\) and 0 otherwise. Thus, the linearized deformation gradient is:
\[ \vect{F} = \vect{I} + \nabla \vect{u} \]

% where \(I\) is the identity matrix, and \(\nabla U\) represents the displacement gradient tensor.

\subsection{Determinant of \(F\) (\(J\))}

% The determinant of \(F\), \(J\), describes the volume change during deformation. For infinitesimally small deformations, \(J\) can be approximated as:
\[ J = \text{det}(\vect{F}) = \text{det}(\vect{I} + \nabla \vect{u}) \]

When the deformation and displacements are infinitesimally small, \(J\) could be approximated as the determinant of an identity matrix plus a small perturbation,
\[ J \approx 1 + \text{Tr}(\nabla \vect{u}) \]

where \(\text{Tr}(\nabla \vect{u})\) is the trace of the displacement gradient, equivalent to the divergence of the displacement field, assuming those higher-order terms in the displacement gradient are negligible.

\subsection{Infinitesimal Strain Tensor (\(E\))}

The Lagrangian strain tensor \({\vect{E}}_{L}\)can be defined in terms of the deformation gradient $\vect{F}$ as:
\[ {\vect{E}}_{L} = \frac{1}{2}({\vect{F}}^T \vect{F} - \vect{I}) \]

For infinitesimallysmall deformations, %substituting the linearized expression for \(F\):
\begin{align*}
{\vect{E}} & = \frac{1}{2}((\vect{I} + \nabla \vect{u})^T(\vect{I} + \nabla \vect{u}) - \vect{I}) \\
\iff E_{ij} & = \frac{1}{2} ((\delta_{ij} + \frac{\partial u_i}{\partial X_j})^T(\delta_{ij} + \frac{\partial u_i}{\partial X_j}) - \delta_{ij}) \\
\iff & = \frac{1}{2} (({\delta_{ij}}^T + {\frac{\partial u_i}{\partial X_j}}^T)(\delta_{ij} + \frac{\partial u_i}{\partial X_j}) - \delta_{ij}) \\
\iff & = \frac{1}{2} (({\delta_{ji}} + {\frac{\partial u_j}{\partial X_i}})(\delta_{ij} + \frac{\partial u_i}{\partial X_j}) - \delta_{ij}) \\
\iff & = \frac{1}{2} ({\delta_{ji}} {\delta_{ij}} + {\delta_{ji}}{\frac{\partial u_i}{\partial X_j}} + \delta_{ij}\frac{\partial u_j}{\partial X_i} +\frac{\partial u_j}{\partial X_i}\frac{\partial u_i}{\partial X_j} - \delta_{ij}) \\
\intertext{Neglecting the higher order term $\frac{\partial u_j}{\partial X_i}\frac{\partial u_i}{\partial X_j}$, and $\delta_{ji}=\delta_{ij}$,}
{\vect{E}} & \approx \frac{1}{2}(\nabla \vect{u} + (\nabla \vect{u})^T)
\end{align*}
This is known as the infinitesimal strain tensor. The same result could be deduced from the Eulerian Strain tensor approach as well.

\newpage

\section{\textbf{Question 2}}
\subsection{Part a}
\begin{align*}
\nabla\phi & =\vect{F} = 
\begin{bmatrix} \cos(\omega t) & \sin(\omega t) & 0 \\
               -\sin(\omega t) & \cos(\omega t) & 0 \\
                0            & 0            & 1+\alpha t \end{bmatrix}\\
\intertext{This deformation gradient could be treated by two separate parts, one is the rotation matrix and other one is the strech matrix. With the property of rotation matrix: ${\vect{Q}}^{-1} ={\vect{Q}}^{T}$}
\nabla \psi &= {\vect{F}}^{-1} = \begin{bmatrix} \cos(\omega t) & -\sin(\omega t) & 0 \\
    \sin(\omega t) & \cos(\omega t) & 0 \\
     0            & 0            & \frac{1}{1+\alpha t} \end{bmatrix}\\
\intertext{Thus, the inverse motion $\vect{X}=\psi(\vect{x},t)$ is given by:}
 X_1 &= \cos(\omega t)x_1 - \sin(\omega t)x_2\\
 X_2 &= \sin(\omega t)x_1 + \cos(\omega t)x_2\\
 X_3 &= \frac{1}{1+\alpha t} x_3
\end{align*}

\subsection{Part b}
The spatial velocity field $\vect{v}(\vect{x},t)$ is given by:
\begin{align*}
\frac {\partial \phi(X)}{\partial t} &= 
    \begin{bmatrix} -\omega \sin(\omega t) & \omega\cos(\omega t) & 0 \\
                   -\omega \cos(\omega t) & -\omega\sin(\omega t) & 0 \\
                    0            & 0            & \alpha \end{bmatrix}\\
\vect{v}(\vect{x},t) &= \frac {\partial \phi(x)}{\partial t} = \frac {\partial \phi(X)}{\partial t} \cdot {\vect{F}}^{-1}\\
&= \begin{bmatrix} -\omega \sin(\omega t) & \omega\cos(\omega t) & 0 \\
    -\omega \cos(\omega t) & -\omega\sin(\omega t) & 0 \\
     0            & 0            & \alpha \end{bmatrix} \cdot \big (\begin{bmatrix} \cos(\omega t) & -\sin(\omega t) & 0 \\
        \sin(\omega t) & \cos(\omega t) & 0 \\
         0            & 0            & \frac{1}{1+\alpha t} \end{bmatrix} \cdot\begin{bmatrix}x_1\\x_2\\x_3\end{bmatrix}\big )\\
&= \begin{bmatrix} \omega x_2 \\ -\omega x_1 \\ \frac{\alpha}{1+\alpha t}x_3 \end{bmatrix}\\
\end{align*}

\subsection{Part c}
Symmetric part of $\nabla \vect{v}$:
\begin{align*}
    \vect{L} &= \begin{bmatrix}
        0&0&0\\
        0&0&0\\
        0&0&\frac{\alpha}{1+\alpha t}\\
    \end{bmatrix}
\end{align*}
Skew symmetric part of $\nabla \vect{v}$:
\begin{align*}
    \vect{W} &= \begin{bmatrix}
        0&-\omega&0\\
        -\omega&0&0\\
        0&0&0\\
    \end{bmatrix}
\end{align*}
Thus, one can conclude that $\vect{L}$ is only related to $\alpha$, and $\vect{W}$ is only related to $\omega$. 



\newpage
\section{\textbf{Question 3}}
By the definition of total/material derivative and partial derivative:
\begin{align*}
D_t f(t,x) &= \frac{\partial f(t,x)}{\partial t} + \frac{\partial f(t,x)}{\partial x}\frac{\partial x}{\partial t}\\
&= {\partial}_t f(t,x) + \nabla f(t,x) \cdot v
\end{align*}
This total time derivative measures the rate of change of a function of $x$ and $t$ following the motion of the material.
The first term indicates the rate of change of $f$ at a point, and the second term (also called convection term) indicates
the effect of that point moving with the material.

\newpage
\section{\textbf{Question 4}}
The main conflicts come up when coupling the fluid and the solid part of a problem since they use different descriptions.
In the Eulerian framework, the motion of the fluid is described from a spatial point of view, focusing on specific points 
in space through which the fluid flows. Variables such as velocity, pressure, and density are expressed as functions of 
fixed spatial positions and time. This approach is natural for fluids, as it allows for the easy handling of the flow's 
continuous deformation and movement without needing to track individual fluid particles. However, the
Eulerian description cannot handle moving boundaries. 
Conversely, the Lagrangian description focuses on individual particles or points within a solid material, tracking their 
positions over time as the material deforms. This method expresses variables like displacement, strain, and stress as 
functions of the material's initial positions and time. The Lagrangian perspective is also natural to solid mechanics, 
where the deformation of the material and the paths of specific material points are of primary interest.
Nevertheless, it is challenging to capture large distortions and topological changes in the domain by Lagrangian description.
In the coupled fluid-structure interaction problem, the deformed structure tends to change
the configuration of the fluid domain too. Thus, we need to resolve the fluid dynamic
equations on a moving (or, time dependent) domain, where requires ALE.
\begin{align*}
\nabla^{\vect{x}} f(\vect{x},t) &= \nabla^{\vect{x}} f(\phi (\vect{X}),t)\\
&= \frac{\partial f(\vect{x},t)}{\partial \phi(\vect{X},t)} \cdot \frac{\partial \phi(\vect{X},t)}{\partial \vect{x}}\\
&= \frac{\partial f(\vect{x},t)}{\partial \phi(\vect{X},t)} \cdot \frac{\partial \phi(\vect{X},t)}{\partial \vect{X}} \cdot \frac{\partial \vect{X}}{\partial \vect{x}}\\
&= \nabla^{\vect{X}} f(\phi (\vect{X}),t) \cdot {\vect{F}}^{-1}\\
&= \hat \nabla \hat f \cdot {\vect{F}}^{-1}
\end{align*}


\newpage
\section{\textbf{Question 5}}
Piola transformation: 
\begin{align*}
\hat {\vect{\sigma}} &= \hat J \vect{\sigma} {\vect{F}}^{-T}\\
\intertext{So,}
\hat {\nabla} \cdot \hat {\vect{\sigma}} &= \hat \nabla \cdot (\hat J \vect{\sigma} \cdot {\vect{F}}^{-T})\\
\intertext{Using index notation:}
% &= \hat J \frac{\partial}{\partial X_i} \hat e_i \cdot (\sigma_{jk} \hat e_j \hat e_k \cdot \frac{\partial X_l}{\partial x_m}\hat e_m \hat e_l) \delta_{km}\\
&= \hat J \frac{\partial}{\partial X_i} \hat e_i \cdot (\sigma_{jk} \frac{\partial X_l}{\partial x_k} \hat e_j \hat e_l)\\
&= \hat J \frac{\partial}{\partial X_i} (\sigma_{jk} \frac{\partial X_l}{\partial x_k} \hat e_l \delta_{ij} \delta_{il})\\
% &= \hat J \frac{\partial}{\partial X_i} \frac{\partial X_i}{\partial x_k} \sigma_{ik} \hat e_i \\
&= \hat J \frac{\partial}{\partial x_k} \sigma_{ik} \hat e_i\\
&= \hat J \nabla \cdot \sigma
\end{align*}

\end{document}