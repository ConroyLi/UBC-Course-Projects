\documentclass[17pt]{extarticle} % The document class with options

\usepackage[margin=1in]{geometry}
\usepackage{newtxtext,newtxmath}
\usepackage[T1]{fontenc}
\usepackage{amsmath}
\usepackage{amsfonts}
\usepackage{microtype}
\usepackage{graphicx}
% chktex-file 3
% chktex-file 18
% chktex-file 36
% chktex-file 44

\begin{document}
\setlength{\parskip}{1em} 
\setlength{\parindent}{0pt}
\newcommand{\vect}[1]{\mathbf{#1}}

\title{MECH 503 LAMMPS Project Report}
\author{Jincong Li \\ 60539939}
\date{\today}
\maketitle

\section*{GN-m}
The Gauss-Newton method is an algorithm used to solve non-linear least squares problems, which often arise in parameter estimation for algebraic models. This method aims to find the parameter values that minimize the sum of squared differences between observed and model-predicted values. The Marquardt modification, also known as the Levenberg-Marquardt algorithm, enhances the Gauss-Newton method by making it more robust and allowing it to converge in situations where Gauss-Newton might fail.

The Gauss-Newton Method
At its core, the Gauss-Newton method iteratively refines the parameter estimates to minimize the discrepancy between observed data and the model. This is particularly suited for problems where the model is a non-linear function of the parameters but linear with respect to the changes in the parameters (linearizable).

Given a set of observations are the independent variables and 
beta represents the parameters to be estimated, the goal is to minimize the sum of squared residuals 


The Gauss-Newton method updates the parameter estimates in each iteration using the formula:

where J is the Jacobian matrix of partial derivatives of the model functions with respect to the parameters 
beta, and r is the vector of residuals between the observed and predicted values.

Marquardt`s Modification (Levenberg-Marquardt Algorithm)
The Levenberg-Marquardt algorithm modifies the Gauss-Newton method by introducing a damping factor 

lambda to improve the convergence properties, especially when the Jacobian matrix 

J is not invertible or near singular. The updated formula for parameter adjustment becomes:

lambda adjusts dynamically during iterations: when 

lambda is high, the algorithm behaves more like gradient descent, taking smaller steps; when 

lambda is low, it behaves more like the Gauss-Newton method, taking larger, more aggressive steps.

The essence of Marquardt's modification is to combine the strong points of both the Gauss-Newton method and gradient descent, allowing for a more versatile and robust approach to parameter estimation in non-linear models. The algorithm iteratively adjusts 

lambda and the parameters 

beta until convergence is reached, typically when the change in the sum of squared residuals is below a certain threshold.

In summary, the Levenberg-Marquardt algorithm is a powerful tool for parameter estimation in non-linear models, providing a compromise between the speed of Gauss-Newton and the stability of gradient descent. It's widely used in various scientific and engineering applications where parameter estimation is crucial.
\section*{CI}
To compute the confidence intervals for the estimated parameters, the approach is based on the statistical properties of the least squares estimators, typically involving the standard errors of these estimators and the assumption that they follow a t-distribution in the context of small sample sizes.

Here’s the general method I outlined for computing the confidence intervals:

Covariance Matrix Calculation:

After obtaining the parameter estimates through the optimization process (e.g., Gauss-Newton method with Marquardt's modification), we calculate the covariance matrix of the parameter estimates. This matrix is derived from the inverse of the Hessian matrix, which, in the context of the Gauss-Newton method, is approximated by 
J TJ, where J is the Jacobian matrix of partial derivatives of the residuals with respect to the parameters.
In cases where J T
J is singular or near-singular (indicating potential issues with model identifiability or data collinearity), we used the pseudo-inverse for stability.
Standard Errors of the Parameter Estimates:

The diagonal elements of the covariance matrix provide the variances of the parameter estimates. Taking the square root of these diagonal elements yields the standard errors of the estimates.
Confidence Interval Calculation:

With the standard errors, we can compute the confidence intervals for the parameters. For a given confidence level (e.g., 95\%), we determine the critical value from the t-distribution, considering the degrees of freedom (which typically depend on the number of data points and the number of parameters).
The confidence interval for each parameter is then calculated as: 
\section*{NMA}

The Nelder-Mead algorithm, also known as the simplex method or downhill simplex method, is a popular heuristic search method used for minimizing an objective function in a multidimensional space. Unlike the Gauss-Newton or Levenberg-Marquardt algorithms, the Nelder-Mead method does not require the objective function to be differentiable, making it particularly useful for optimizing non-linear and non-smooth functions. It's commonly applied in parameter estimation tasks for algebraic models where gradient-based methods might struggle due to discontinuities or the absence of derivatives.

How the Nelder-Mead Algorithm Works
The algorithm operates using a simplex, which in two dimensions is a triangle, but in higher dimensions, it generalizes to a polytope (a geometric object with flat sides). The simplex represents a set of candidate solutions that are iteratively updated to explore the space of parameters to find the minimum of the objective function. The method involves the following key steps:

Initialization: Start with an initial simplex, which is typically generated by providing a starting point and then adding points that are slightly offset in the direction of each dimension.

Evaluation: Calculate the objective function value (e.g., the error or cost) at each vertex of the simplex.

Transformation: Update the simplex through a series of transformations: reflection, expansion, contraction, and shrinkage, based on the comparison of the objective function values at the vertices.

Reflection: Reflects the worst point (the point with the highest objective function value) across the centroid of the remaining points, possibly replacing the worst point if the reflected point is better.

Expansion: If the reflection is successful and finds a new best point, the algorithm tries to take a larger step in the same direction to see if it can improve further.

Contraction: If reflection does not result in improvement, the algorithm performs a contraction to move the worst point closer to the centroid, hoping to find a better position within a tighter search space.

Shrinkage: If neither reflection nor contraction significantly improves the worst point's position, the algorithm reduces the size of the simplex around the best point, shrinking the search space in hopes of finer exploration.

Termination: These steps are repeated until a termination criterion is met, which could be a small enough change in the objective function value, a sufficiently small simplex size, or a maximum number of iterations.
\section*{GN-m}
\section*{GN-m}


\section*{bootstrap method}

The bootstrap method is a powerful statistical tool used to estimate the distribution of a statistic (like the mean or variance) from a set of data. It involves repeatedly sampling from the data set with replacement to create "bootstrap samples," and then computing the statistic of interest for each sample. This generates a distribution of the statistic, from which confidence intervals and other properties can be derived. The method is especially useful when the theoretical distribution of the statistic is unknown or difficult to derive.

Key Steps in the Bootstrap Method:
Resampling: From your original data set of size
N, randomly sample 
N observations with replacement to create a bootstrap sample. This sample is the same size as the original but may contain duplicates.

Computation: Calculate the statistic of interest (e.g., mean, variance, parameter estimates) for the bootstrap sample.

Repetition: Repeat steps 1 and 2 many times (e.g., 1000 or more iterations) to create a distribution of the statistic.

Estimation: Use the distribution obtained from step 3 to estimate the confidence intervals or variance of the statistic.

Implementation for Parameter Estimation:
In the context of parameter estimation for a given model (like the algebraic model we discussed), the bootstrap method can estimate the distribution of the parameters and derive confidence intervals. Here's a simplified overview of how it was implemented:

Bootstrap Sampling: For each iteration, a new dataset is created by randomly sampling the original data with replacement. This simulates drawing new data from the underlying population.

Parameter Estimation: For each bootstrap sample, the Nelder-Mead Simplex algorithm (or any other optimization method) is used to estimate the model parameters that minimize the difference between the model's predictions and the actual data.

Distribution of Estimates: After many iterations, this process produces a distribution of estimated values for each parameter.

Confidence Interval Calculation: The confidence intervals are then calculated from these distributions. For a 95\% confidence interval, the 2.5th and 97.5th percentiles of the estimated values for each parameter are used.

Practical Considerations:
Number of Iterations: More iterations increase the accuracy of the confidence interval estimates but also require more computational resources.

Random Sampling with Replacement: Ensures that each bootstrap sample is an independent and identically distributed sample of the data, mirroring the sampling variability in the original dataset.

Applicability: While bootstrap is widely applicable, its validity depends on the data being representative of the population. In cases where the original sample may not represent the population well, bootstrap estimates may be biased.

This method's strength lies in its simplicity and applicability to a wide range of problems, especially in situations where traditional parametric inference is challenging.



\end{document}