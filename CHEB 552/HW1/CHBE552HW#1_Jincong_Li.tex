\documentclass[a4paper,12pt]{article} % The document class with options

\usepackage[T1]{fontenc}
\usepackage{amsmath}
\usepackage{amsfonts}
\usepackage{microtype}

\begin{document}
%\setlength{\parskip}{1em} 
%\setlength{\parindent}{0pt}
\newcommand{\vect}[1]{\mathbf{#1}}

\title{CHBE 552 Problem Set 1}
\author{Jincong Li \\ 60539939}
\date{Jan 24th}
\maketitle

\section{\textbf{Question 1}}

\subsection{Part a}

\begin{align*}
F( \vect{x} )  &= (x_1 - 2)^4 + (x_1-2x_2)^2 \\
\partial F_{x_1} &= 2 x_{1} - 4 x_{2} + 4 \left(x_{1} - 2\right)^{3} \\
\partial F_{x_2} &= -4x_1 + 8x_2\\
\intertext{By solving each partial derivative (set equal to zero), the stationaty point is obtained to be}
\begin{pmatrix} 
{x_1}^* \\ {x_2}^* 
\end{pmatrix}
&= 
\begin{pmatrix} 
2\\1
\end{pmatrix}
\\
\intertext{The Hessian matrix is determined to be}
H_a &= \left[\begin{matrix}12 \left(x_{1} - 2\right)^{2} + 2 & -4\\-4 & 8\end{matrix}\right]\\
\intertext{Then, the Hessian matrix at the stationary point is evaluated to be}
H_{a,eva.}&=\left[\begin{matrix}2 & -4\\-4 & 8\end{matrix}\right]\\
\intertext{The eigen values of the evaluated Hessian matrix are computed to be $0, 10$, thus that stationary point is determined to be a local minimum since the function is Convex at this point}
\end{align*}

\subsection{Part b}
\begin{align*}
F( \vect{x} )  &= 2x_1^3 + x_2^2 + x_1^2x_2^2 + 4x_1x_2 + 3 \\
\partial F_{x_1} &= 6x_1^2 + 2x_1x_2^2 + 4x_2 \\
\partial F_{x_2} &= 2x_1^2x_2 + 4x_1 + 2x_2 
\intertext{By solving each partial derivative (set equal to zero), the stationaty point is obtained to be}
\begin{pmatrix} 
{x_1}^* \\ {x_2}^* 
\end{pmatrix}
& =
\begin{pmatrix} 
0\\0
\end{pmatrix}
\\
\intertext{The Hessian matrix is determined to be}
H_b &=\left[\begin{matrix}12 x_{1} + 2 x_{2}^{2} & 4 x_{1} x_{2} + 4\\4 x_{1} x_{2} + 4 & 2 x_{1}^{2} + 2\end{matrix}\right]\\
\intertext{Then, the Hessian matrix at the stationary point is evaluated to be}
H_{b,eva.}&=\left[\begin{matrix}0 & 4\\4 & 2\end{matrix}\right]\\
\intertext{The eigen values of the evaluated Hessian matrix are computed to be $2, -16$, thus that stationary point is determined to be a saddle point}
\end{align*}

\subsection{Part c}
\begin{align*}
F(x)  &= 12x^5 - 45x^4 + 40x^3 + 5 \\
F^{\prime} (x) &= 60x^4 - 180x^3 + 120x^2 \\
\intertext{By solving the first derivative (set equal to zero), the stationaty points are obtained to be}
x^* &= 0,1,2\\
\intertext{The second derivative is determined to be}
F^{\prime\prime} (x)&=240 x^{3} - 540 x^{2} + 240 x\\
\intertext{Then, the second derivative at the stationary point is evaluated to be}
F^{\prime\prime}(x^*)&=\left[ 0, \  -60, \  240\right]\\
\intertext{By checking the eigen values of the second derivative, those stationary points are determined to be a saddle point($x=0$),a local maximun($x=1$), and a local minimum($x=2$)}
\end{align*}

\section{\textbf{Question 2}}
\subsection{Part a}

\begin{equation*}
\begin{split}    
c &= \frac{58.0555555555556}{T^{0.3017}} + \frac{70.8612328738115 T^{0.4925}}{q}\\
  & \quad + \frac{2.82512546852169 T^{0.7952}}{q} + \frac{19.8593073947553}{q^{0.1899}} \\
  & \quad + 8.48356524677597 \cdot 10^{-5} q^{0.671} + 13.9\\
  & \quad + \frac{0.000332994515384452 \cdot \left(1700.0 T + 162.162162162162 q\right)}{q}\\
\end{split}
\end{equation*}
\text{the partial derivatives are found to be}
\begin{align*}
\partial c_q &= - \frac{22.531393600576 T^{0.4925}}{q^{3}} \\
\partial c_T &= - \frac{17.5153611111111}{T^{1.3017}} + \frac{5.54835567414184}{T^{0.5075} q^{2}}\\
\intertext{Since the Hessian matrix of the cost function could be obtained though its complicated and long}
\intertext{So with the implementation of Newton's Method, the optimum tanker size and refinery size are computed to be}
q &= 174833 \text{bbl/day} \\
T &= 484726 \text{kL}
\intertext{And the minimun cost of oil is then computed to be}
c_{min} &= 17.8226 
\end{align*}



\subsection{Part b}
\begin{align*}
    \intertext{The profit is found to be}
    P &= 50*Yp - \text{cost of additive of A per mol} - \text{cost of steam per mol}\\
      &= - 2000000 S^{2} + 0.005 S x_{a} + 0.047 S - 20 x_{a}^{2} + 5.0 x_{a} + 1.0\\
    \intertext{And the corresponding Hessian matrix are calculated to be}
    H_P &= \left[\begin{matrix}-40 & 0.005\\0.005 & -4000000\end{matrix}\right]\\
    \intertext{The eigen values are approximated to be $-40$ and $-4000000$, 
    which are both negative. Therefore, the profit function is concave. 
    This conclusion makes sense since it is rational to have a optimum input of material A and steam
    such that profit is maximized}
\end{align*}
    


\subsection(Part c)
\begin{align*}1
    232131
\end{align*}
\end{document}