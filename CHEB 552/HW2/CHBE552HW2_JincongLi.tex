\documentclass[a4paper,12pt]{article} % The document class with options

\usepackage[T1]{fontenc}
\usepackage{amsmath}
\usepackage{amsfonts}
\usepackage{microtype}
% chktex-file 3

\begin{document}
%\setlength{\parskip}{1em} 
%\setlength{\parindent}{0pt}
\newcommand{\vect}[1]{\mathbf{#1}}

\title{CHBE 552 Problem Set 1}
\author{Jincong Li \\ 60539939}
\date{Feb 24th}
\maketitle

\section*{\textbf{Question 1}}

\subsection*{Part a}
\begin{align*}
    \vect{F_1} &= x_{1}^{2} + x_{1} x_{2} + x_{1} x_{3} + x_{1} x_{4} + x_{1}\\
    & + x_{2}^{2} + x_{2} x_{3} + x_{2} x_{4} + x_{2} + x_{3}^{2} + x_{3} x_{4}\\
    & + x_{3} + x_{4}^{2} + x_{4} + 1 \\
    \vect{H_{F_1}} &= \left[\begin{matrix}2 & 1 & 1 & 1\\1 & 2 & 1 & 1\\1 & 1 & 2 & 1\\1 & 1 & 1 & 2\end{matrix}\right] \\
    \vect{F_2} &= 8 x_{1}^{2} + 4 x_{1} x_{2} + 5 x_{2}^{2} \\
    \vect{H_{F_2}} &= \left[\begin{matrix}16 & 4\\4 & 10\end{matrix}\right] \\
    \intertext{By implementating Newton's method, the corresponding minimums are found to be:}
    \vect{x_{F1}} &= \left[\begin{matrix}-0.2\\-0.2\\-0.2\\-0.2\end{matrix}\right] \text{or} \  \left[\begin{matrix}-0.2\\-0.2\\-0.2\\-0.2\end{matrix}\right]\\
    \vect{x_{F2}} &= \left[\begin{matrix}0\\0\end{matrix}\right]
    \intertext{Note that for $\vect{F_1}$, two initial guesses are provided the value of $\vect{x_{F1}}$ shown above
    are derived from thoes two initial guesses and they are actually the same.}
\end{align*}

\subsection*{Part b}
\begin{align*}
    \vect{F} &= 4 \left(x_{1} - 5\right)^{2} + \left(x_{2} - 6\right)^{2} \\
    \nabla \vect{F}&= \left[ 8 x_{1} - 40, \  2 x_{2} - 12\right]\\
    \intertext{By implementating Fletcher Reeves's method, the minimum is found to be:}
    \vect{x} &= \left[\begin{matrix}5.11\\2.48\end{matrix}\right]
\end{align*}

\subsection*{Part c}
\begin{align*}
    \vect{F} &= 2 x_{1}^{2} + 2 x_{1} x_{2} + x_{1} + x_{2}^{2} - x_{2} \\
    \nabla \vect{F}&= \left[ 4 x_{1} + 2 x_{2} + 1, \  2 x_{1} + 2 x_{2} - 1\right]\\
    \intertext{By implementating DFP method, the minimum is found to be:}
    \vect{x} &= \left[\begin{matrix}-0.65\\0.98\end{matrix}\right]
\end{align*}

\section*{\textbf{Question 2}}

\subsection*{Part a}
\begin{align*}
    F &= x_{1}^{2} + 10 x_{1} + x_{2}^{2} + 20 x_{2} + 25 \\
    L &= \lambda \left(x_{1} + x_{2}\right) + x_{1}^{2} + 10 x_{1} + x_{2}^{2} + 20 x_{2} + 25 \\
    \partial F_{x_1} &= \lambda + 2 x_{1} + 10 \\
    \partial F_{x_2} &= \lambda + 2 x_{2} + 20 \\
    \partial F_{\lambda} &= x_{1} + x_{2} \\
    \intertext{By setting those partial derivative equations to zero, the optimum values are found to be:}
    x_{1,opt} &= \frac{5}{2} \\
    x_{2,opt} &= - \frac{5}{2} \\
    \lambda_{opt} &= -15 \\
    \intertext{At this optimum point the original function is evaluated to be:}
    F_{opt} &= \frac{25}{2} \\
    \intertext{For a sensitivity test the constraint condition is changed to $x_1 + x_2 = 0.01$, then the previous
    steps are repeated:}
    L_{new} &= \lambda \left(x_{1} + x_{2} - 0.01\right) + x_{1}^{2} + 10 x_{1} + x_{2}^{2} + 20 x_{2} + 25 \\
    \partial F_{x_1} &= \lambda + 2 x_{1} + 10 \\
    \partial F_{x_2} &= \lambda + 2 x_{2} + 20 \\
    \partial F_{\lambda} &= x_{1} + x_{2} - 0.01 \\
    x_{1,opt} &= 2.505 \\
    x_{2,opt} &= -2.495 \\
    \lambda_{opt} &= -15.01\\
    F_{new,opt} &= 12.65005 \\
    \intertext{Thus, the increment of the function value is computed to be}
    \Delta F &= 0.15
\end{align*}

\subsection*{Part b}
\begin{align*}
    F &= - \pi x_{1}^{2} x_{2} \\
    L &= \lambda \left(2 \pi x_{1}^{2} + 2 \pi x_{1} x_{2} - 24 \pi\right) - \pi x_{1}^{2} x_{2} \\
    \partial F_{x_1} &= \lambda \left(4 \pi x_{1} + 2 \pi x_{2}\right) - 2 \pi x_{1} x_{2} \\
    \partial F_{x_2} &= 2 \pi \lambda x_{1} - \pi x_{1}^{2} \\
    \partial F_{\lambda} &= 2 \pi x_{1}^{2} + 2 \pi x_{1} x_{2} - 24 \pi \\
    \text{Solution} &= \left[ \left( 2, \  4\right)\right]
\end{align*}

\section*{\textbf{Question 3}}
\section*{\textbf{Question 4}}
\begin{align*}
    F &= y^{2} + \left(x - 2\right)^{2} + \left(z - 1\right)^{2} \\
    L &= \lambda \left(- 4 x^{2} - 2 y^{2} + z^{2}\right) + y^{2} + \left(x - 2\right)^{2} + \left(z - 1\right)^{2} \\
    \partial F_{x} &= - 8 \lambda x + 2 x - 4 \\
    \partial F_{y} &= - 4 \lambda y + 2 y \\
    \partial F_{z} &= 2 \lambda z + 2 z - 2 \\
    \partial F_{\lambda} &= - 4 x^{2} - 2 y^{2} + z^{2} \\
    \vect{x} &=   \left( \frac{4}{5}, \  0, \  \frac{8}{5}\right) \\
    \intertext{With $\lambda = - \frac{3}{8}$}
\end{align*}

\section*{\textbf{Question 5}}
\subsection*{Part a}
\begin{align*}
    F &= x_{1}^{2} - 14 x_{1} + x_{2}^{2} - 6 x_{2} - 7 \\
    L &= \lambda_{1} \left(x_{1} + x_{2} - 2\right) + \lambda_{2} \left(x_{1} + 2 x_{2} - 3\right) + x_{1}^{2} - 14 x_{1} + x_{2}^{2} - 6 x_{2} - 7 \\
    \partial F_{x} &= \lambda_{1} + \lambda_{2} + 2 x_{1} - 14 \\
    \partial F_{y} &= \lambda_{1} + 2 \lambda_{2} + 2 x_{2} - 6 \\
    \partial F_{\lambda_1} &= x_{1} + x_{2} - 2 \\
    \partial F_{\lambda_2} &= x_{1} + 2 x_{2} - 3 \\
    \vect{x} &= \left\{ x_{1} : 1, \  x_{2} : 1\right\} \\
    \intertext{With}
    \lambda_{1,2} &= \left\{ \lambda_{1} : 20, \  \lambda_{2} : -8\right\}
    \intertext{And the minimum objective function value is computed to be $-25$.}
\end{align*}

\subsection*{Part b}
\begin{align*}
    F &= x_{1}^{2} + 2 \left(x_{2} + 1\right)^{2} \\
    L &= \lambda_{1} \left(- x_{1} + x_{2} - 2\right) + \lambda_{2} \left(- x_{1} - x_{2} - 1\right) + x_{1}^{2} + 2 \left(x_{2} + 1\right)^{2} \\
    \partial F_{x} &= - \lambda_{1} - \lambda_{2} + 2 x_{1} \\
    \partial F_{y} &= \lambda_{1} - \lambda_{2} + 4 x_{2} + 4\\
    \partial F_{\lambda_1} &= - x_{1} + x_{2} - 2\\
    \partial F_{\lambda_2} &= - x_{1} - x_{2} - 1\\
    \vect{x} &= \left\{x_{1} : - \frac{3}{2}, \  x_{2} : \frac{1}{2}\right\} \\
    \intertext{With}
    \lambda_{1,2} &= \left\{ \lambda_{1} : - \frac{9}{2}, \  \lambda_{2} : \frac{3}{2}\right\}
    \intertext{And the minimum objective function value is computed to be $\frac{27}{4}$.}
\end{align*}
\end{document}

\section*{\textbf{Question 6}}