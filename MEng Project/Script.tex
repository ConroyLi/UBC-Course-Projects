\documentclass[17pt]{extarticle} % The document class with options

\usepackage[margin=1in]{geometry}
\usepackage{newtxtext,newtxmath}
%\usepackage[T1]{fontenc}
\usepackage{amsmath}
\usepackage{amsfonts}
\usepackage{microtype}
% chktex-file 3
% chktex-file 36

\begin{document}
\setlength{\parskip}{1em} 
\setlength{\parindent}{0pt}
hi Dr. jasmine and Dr. hannan, I am so glad to have both of you here for my presentation.
so the topic of the meng project is the verificattion and validation of the Numerical
modelling the DTMB 5415 ship model

i will start by introducing this project, then i will talk about the methodology I used, 
and later I will present the results and some discussion on the result.

Start by defining the problem this project is aiming to solve, this work numerically 
model the DTMB 5415 ship in the simflow and test it in the head wave condition. The
result from the numerical simation will be compared to the experimental data report in a published
paper from the Journal named ocean engineering. The main result of the simulation that this work 
investigated in the vertical shear force acting on the ship. Also, since this is numerical simulation,
mesh convergence tests are conducted in three difference mesh condition to ensure the accuracy of the result.

The model employed in this study is the David Taylor Model Basin (DTMB 5415), which serves as a 
critical benchmark in naval hydrodynamic research. After its concept was developed in 1980s, thie ship model 
has been extensively utilized in both experimental and computational studies due to its representative features of modern naval vessels.
Those features are the transom stern and the sonar dome.

Here are the dimentions of this DTMB ship, notice that for the ease of modeling and testing, a scaled model is often used.
For this work, the 1 by 24 model is used, for the paper that has the experimental data, they used a 1/51 model.
And here are the scaled dimension, i know there should be more dimention data availale for this ship and they are well documented,
but I only present the particulars that matters in my study and those numbers are important when it comes to compute the non-dimensional result.

the software or the tool utilized in this project to conduct the simualtion is the simflow, since it is unpublished and for researching 
purpose, I dont really know too much of it. but it solves NS equations just like other CFD softwares. And it can
output the integrated force on a defined surface and the timehistory of it.

The computational domain is constructed in dimensions of 20 meters by 6 meters by 6 meters in xyz coordinates 
system as shown below as well as the boundaris defined like this.

The inlet boundary \(\Gamma_{\text{in}}\) is exposed to the free surface waves,The side boundaries \(\Gamma_{\text{side}}\) and the bottom boundary 
\(\Gamma_{\text{bottom}}\) are modeled with slip boundary conditions, but this no perpendicular velocities.
No slip boudary is satified on the surface of the ship \(\Gamma_{\text{ship}}\). 
Atmospheric condition of \(p = 0\) is satisfied at the top boundary \(\Gamma_{\text{top}}\).

Second order Stokes' waves is utilized in this scenario since the second-order Stokes waves offer significant advantages over Airy linear 
waves, particularly in their ability to account for nonlinear effects,  accurately represent wave 
steepness, and predict higher harmonics, which are crucial in realistic wave modeling.

for better represents the wave condtion in the experiment, the same dimensional wave conditions are set according to the paper.

The computational domain is meshed to be generally coarse to reduce the computational cost.
but the mesh is refined in the region through which the waves 
propagate to better simulate the wave motion. Additionally, the mesh is further refined in the area surrounding 
the ship model to accurately capture the forces acting on the vessel.Eventually, this mesh has around 2 millions of tetrahedra elements.
And this refinement is justified in the mesh convergence tests which will be discussed later.

parameters.

the vertical force acting on the ship will be record in the .oisd files for each time step.
and the date will be analyzed in a MATLAB scipt.

So, here is the reult section. the upper one is the time history of the VSF in my simulation as well as
the wave. the lower one is the  experimental data. they are both trimed to have a 20 second time period
for better comparison.  We can observe the perfect sinusoidal response of the VSF. But for quantatatively 
compare the result, we need to look into the non-dimensional VSF, which can be computed from this equation.
F is the recoreded net force, density of the water, gavity, length and width of the ship model. A here is the
amplitude of the wave. so according to the dimensions of the model that I introduced previouslt, the non-dimensional
VSF in the reference paper is computed to  be 0.0281, and my result is 0.0272. And the perventage error is 
3.36\%.

And here is the result of the mesh convergence test. As i mentioned, the software that I used to generate the mesh it
Gmsh. In Gmsh, I could use the size field feature to set the local size of the elements in a defined region, Additional to this,
I could use the transfinite curve feature to defined the number of elements on the ship model, and this feature will
graduatially refine the mesh when it comes closer to the ship, just as shown in the 3D view of the mesh. So, by
adjusting the number of elements on those curves, I managed to have the total number of elements to range
from  1.6 mil to 2.1mil. And we can see that the error decreased significantly. However, I realized that this
range is too small for a typical mesh convergence test, it should be across one or two order of maginitude. The reason
that Im not testing for less elements is that, i found if the number of elements near the ship model is lower than
a certain value, the result will be unphyical and might break the simualtion. And if the total number of elements
exceed the 2.1 million, it will cause the gmsh stop responding when generate the mesh since this step is done on my local machine.

So, from the result section, I can tell that the numerical simulations accurately captured vertical shear forces,
showing strong agreement with existing experimental data.The results of this study demonstrate the effectiveness of using simflow to simulate the
hydrodynamic performance of the DTMB 5415 ship model under head wave conditions. Meanwhile, the mesh convergence
test found that mesh refinement near the ship model significantly improves the accuracy of
the simulation results, highlighting the importance of precise mesh configuration in capturing
complex hydrodynamic phenomena.

The findings of this study have substantial implications for both theory and practice in naval
architecture and engineering. By confirming simflow’s capability to accurately model ship
performance under head wave conditions, this research supports the broader adoption of
Computational Fluid Dynamics (CFD) tools in the design and analysis of naval vessels. The
validated numerical framework developed in this study can be applied to other ship models and
wave conditions, potentially enhancing the predictive capabilities of CFD across various naval
applications.

Despite the positive outcomes, this study has several limitations. First, the analysis was restricted
to head wave conditions, and the results may not be directly applicable to other wave orientations,
such as beam or following seas. since the result have confirm the simflow is a good software to simulate waves
, we could investigated other scenarios, even for waves that comes from differnce angles and speed.

Also, for saveing the computational cost, the mesh is set to be generally coarse and the size of the domain
is restricred in this study, if possible in the future, one can have a larger domain to represent the real sea
condition and finer mesh in the water domain as well as on the water-air interface, the result would be more,
convincing. This will also largrly affect the way that one could conduct the MCT.

Futhermore, in this study the ship model is stationary, it only feels the loads but not moving in any direction.
and I have confirm the capability of simflow to solve FSI problems at some stages of my project, however, due to
the lack of expermental data, it is not validated yet. But this could be a really good direction to investigate.

Also, in the reference paper, they also investigated the damaged ship model, which is rapid-emerging topic at this moment,
this study could be extend to a damaged ship model as well, just need to modify the ship model a little bit.





\end{document}