\documentclass[17pt]{extarticle} % The document class with options

\usepackage[margin=1in]{geometry}
\usepackage{newtxtext,newtxmath}
%\usepackage[T1]{fontenc}
\usepackage{amsmath}
\usepackage{amsfonts}
\usepackage{microtype}
% chktex-file 3
% chktex-file 36

\begin{document}
\setlength{\parskip}{1em} 
\setlength{\parindent}{0pt}
hi Dr. jasmine and Dr. hannan, I am so glad to have both of you here for my presentation.
so the topic of the meng project is the verificattion and validation of the Numerical
modelling the DTMB 5415 ship model

i will start by introducing this project, then i will talk about the methodology I used, and later I will present the results and the Discussion

Start by defining the problem this project is aiming to solve, this work numerically 
model the DTMB 5415 ship in the simflow and test it in the head wave condition. The
result from the numerical simation will be compared to the experimental data report in a published
paper from the Journal named ocean engineering. The main result of the simulation that this work 
investigated in the vertical shear force acting on the ship. Also, since this is numerical simulation,
mesh convergence tests are conducted in three difference mesh condition to ensure the accuracy of the result.

The model employed in this study is the David Taylor Model Basin (DTMB 5415), which serves as a 
critical benchmark in naval hydrodynamic research. After its concept was developed in 1980s, thie ship model 
has been extensively utilized in both experimental and computational studies due to its representative features of modern naval vessels.
Those features are the transom stern and the sonar dome.

Here are the dimentions of this DTMB ship, notice that for the ease of modeling and testing, a scaled model is often used.
For this work, the 1 by 24 model is used, for the paper that has the experimental data, they used a 1/51 model.
And here are the scaled dimension, i know there should be more dimention data availale for this ship and they are well documented,
but I only present the particulars that matters in my study and those numbers are important when it comes to compute the non-dimensional result.

the software or the tool utilized in this project to conduct the simualtion is the simflow, since it is unpublished and for researching 
purpose, I dont really know too much of it. but it solves NS equations just like other CFD softwares. And it can
output the integrated force on a defined surface and the timehistory of it.

The computational domain is constructed in dimensions of 20 meters by 6 meters by 6 meters in xyz coordinates 
system as shown below as well as the boundaris defined like this.

The inlet boundary \(\Gamma_{\text{in}}\) is exposed to the free surface waves,The side boundaries \(\Gamma_{\text{side}}\) and the bottom boundary 
\(\Gamma_{\text{bottom}}\) are modeled with slip boundary conditions, but this no perpendicular velocities.
No slip boudary is satified on the surface of the ship \(\Gamma_{\text{ship}}\). 
Atmospheric condition of \(p = 0\) is satisfied at the top boundary \(\Gamma_{\text{top}}\).

Second order Stokes' waves is utilized in this scenario since the second-order Stokes waves offer significant advantages over Airy linear 
waves, particularly in their ability to account for nonlinear effects,  accurately represent wave 
steepness, and predict higher harmonics, which are crucial in realistic wave modeling.

for better represents the wave condtion in the experiment, the same dimensional wave conditions are set according to the paper.

The computational domain is meshed to be generally coarse to reduce the computational cost.
but the mesh is refined in the region through which the waves 
propagate to better simulate the wave motion. Additionally, the mesh is further refined in the area surrounding 
the ship model to accurately capture the forces acting on the vessel.Eventually, this mesh has around 2 millions of tetrahedra elements.
And this refinement is justified in the mesh convergence tests which will be discussed later.

parameters.

the vertical force acting on the ship will be record in the .oisd files for each time step.
and the date will be analyzed in a MATLAB scipt.


\end{document}