\documentclass[a4paper,12pt]{article} % The document class with options

\usepackage[T1]{fontenc}
\usepackage{amsmath}
\usepackage{amsfonts}
\usepackage{microtype}
\usepackage{setspace}
% chktex-file 1
% chktex-file 3
% chktex-file 8
% chktex-file 13
\begin{document}
\setlength{\parskip}{1em} 
\setlength{\parindent}{0pt}
\doublespacing

\title{MECH 570C Course Project Proposal
Slamming Load on Ships}
\author{Jincong Li 60539939}
\date{\today}
\maketitle

% \section*{Introduction}
Slamming loads on ships are critical phenomena in naval architecture and marine engineering, 
referring to extreme forces on a ship's hull during high-speed water surface impacts or 
rough sea conditions. These events can cause substantial structural damage, threatening 
the ship's integrity and safety. This project aims to numerically simulate ship-water surface
interactions, focusing on stress distribution and deformation under various conditions.

%\section*{Problem Definition}
The project seeks to simulate and analyze the loads and stresses on ships' fronts when 
encountering large waves and unexpected vertical motions. Such conditions have been reported 
to cause significant stresses, posing a substantial threat to ship safety. Understanding 
these peak loads is crucial for designing resilient ship structures.

%\section*{Importance of the Study}
Investigating the slamming load is vital for societal and practical reasons. Ensuring 
the structural integrity of ships in extreme weather conditions is paramount for the 
safety of cargo and crew. This research will contribute to safer, more robust ship designs 
by identifying peak load conditions and structural deformation responses.

%\section*{State-of-the-Art}
The state-of-the-art in slamming load analysis involves advanced numerical simulations 
that integrate fluid dynamics and structural mechanics. This project will build on current 
methodologies using Computational Fluid Dynamics (CFD) and Fluid-Structure Interaction (FSI) 
technology to accurately predict the complex interactions between waves and ship structures.

%\section*{Methodology}
This project will employ Fluid-Structure Interaction (FSI) technology, utilizing the 
Navier-Stokes equations for fluid dynamics and modeling the ship structure as an elastic body. 
An Arbitrary-Lagrangian-Eulerian (ALE) mesh will be used to dynamically couple the fluid and 
structural domains, accurately simulating their interaction.

%\section*{Expected Outcomes}
The project aims to produce detailed simulations that reveal stress distribution and 
deformation patterns under slamming load conditions. These findings will identify maximum s
tress values and their occurrence conditions, offering insights into necessary structural 
resilience against extreme sea conditions.

%\section*{Tentative Timeline and Milestones}
I will start reviewing literatures related to this topic from now to the midterm presentation.
And then start working on specific tasks that will be identified later untill the final presentation.
Then compile the final report as well as the final presentation.

\end{document}