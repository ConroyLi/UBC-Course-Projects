\documentclass[a4paper,11pt]{article} % The document class with options

\usepackage[margin=1in]{geometry}
\usepackage{newtxtext,newtxmath}
\usepackage[T1]{fontenc}
\usepackage{amsmath}
\usepackage{amsfonts}
\usepackage{microtype}
% chktex-file 3
% chktex-file 36

\begin{document}
\setlength{\parskip}{1em} 
\setlength{\parindent}{0pt}
\newcommand{\vect}[1]{\mathbf{#1}}

\title{MATH 521 Project Proposal}
\author{Jincong Li 60539939}
\date{Mar 2nd}
\maketitle

\section{Introduction}
This project proposes a comprehensive look into Weak Galerkin (WG) finite element 
method for the Navier–Stokes equations. It aims to study another finite element method other than the topics covered in the course but with 
same approach of exploring. The project will delve into the methodology, findings, 
and implications of applying the Weak Galerkin (WG) Finite Element Method to solve 
the Navier-Stokes equations, which are fundamental in describing fluid motion as well as in 
more advanced fields, such as Fluid Structure Interaction (FSI).

\section{Objectives}
The objectives of this project follows the general structure of the course material as how we explore a FEM.
\begin{itemize}
    \item Understand the WG Finite Element Method's development (scheme) and its application to the Navier-Stokes equations.
    \item Discuss the existence and uniqueness of the WG Finite Element Method's solution.
    \item Perform the error estimate and analysis of the WG Finite Element Method.
    \item Evaluate the method's effectiveness (convergence rate), and computational efficiency.
    % \item Explore potential applications in complex fluid dynamics problems.
\end{itemize}

\section{Methodology}
The project will start from reviewing relavant articles and then presenting a summary with personal understanding of the article's contributions. 
% a critical analysis of  its methodology (mathematical formulation, error estimates, numerical experiments), 
Moreover, a comparison with the finite element methods covered in course material could be conducted and analyzed.

\section{Expected Outcomes}
The project aims to clarify the WG Finite Element Method's contributions to computational 
fluid dynamics, discussing its strengths, limitations, and future research directions. 
It will also provide insights into the method's engineering applications, such as FSI mentioned previously.

\end{document}