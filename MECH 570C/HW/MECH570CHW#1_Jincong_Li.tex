\documentclass[a4paper,12pt]{article} % The document class with options

\usepackage[T1]{fontenc}
\usepackage{amsmath}
\usepackage{amsfonts}
\usepackage{microtype}

\begin{document}
\setlength{\parskip}{1em} 
\setlength{\parindent}{0pt}
\newcommand{\vect}[1]{\mathbf{#1}}

\title{MECH 570C Assigment 1}

\section{\textbf{Question 1}}


One example that I could come up with in practical engineering scenario is the wind turbine. The idea comes from a project I carried on previously,  which is controlling an offshore wind farm. The interaction between the upwind flow (incoming wind) and the turbine blades is a classic example of FSI. As wind flows over the blades, it induces forces on the blades and  causes the blades to move, and these movements, in turn, alter the flow of the wind. This dynamic interaction is critical to the turbine's efficiency and structural integrity.

In this process, classical dynamic analysis of fluid and structure are both significantly involed. Also, there are maybe some coupling effect happens between,which will generate complexity in this problem.

Some key non-dimensional parameters might be included in this problem are: Reynolds Number, Strouhal Number, and Froude Number. Re is presented here since flow is involved, obviously. St is due to the existence of vortex shedding and instability issues. Fr is included here in a sense that the sacle of wind turbine (structure) is large, and the gravity is playing a crucial role in this problem.


\newpage

\section{\textbf{Question 2}}
$\vect{r_b[B]}+\vect{r_s[\partial B]}$

\end{document}