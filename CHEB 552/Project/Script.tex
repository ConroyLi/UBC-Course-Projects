\documentclass[a4paper,11pt]{article} % The document class with options

\usepackage[margin=1in]{geometry}
\usepackage{newtxtext,newtxmath}
\usepackage[T1]{fontenc}
\usepackage{amsmath}
\usepackage{amsfonts}
\usepackage{microtype}
% chktex-file 3
% chktex-file 36
\begin{document}
\setlength{\parskip}{1em} 
\setlength{\parindent}{0pt}

\section*{PE}

Parameter estimation in algebraic models is a fundamental process in various fields such as engineering,
economics, physics, and biology. It involves determining the values of parameters within mathematical 
models that best fit a set of observed data. These models are algebraic in nature, meaning they are 
composed of equations using algebraic operations. The aim is to make the model accurately represent 
the real-world process or system being studied.

So, in general The goal of parameter estimation is to find the set of parameters 
that minimizes the difference between the observed data and the model's 
predictions. These differences are often quantified using a cost function, 
typically the sum of squared residuals.

\section*{SM}
It is designed to solve optimization problems where the objective function and the constraints are 
linear. Although primarily used for linear problems, the Simplex method can also be adapted or serve 
as a foundation for approaches that handle certain non-linear parameter estimation problems in 
algebraic models, especially when linear approximations are feasible or when dealing with piecewise-linear
models.

\textbf{Basic Concept}:

The core idea behind the Simplex method is to navigate the vertices of the feasible region defined by the linear 
constraints of an optimization problem to find the optimum value of a linear objective function.

\section*{GN}

The Gauss-Newton algorithm is a method used for solving non-linear least squares problems,
It's particularly well-suited for situations where the model is a non-linear function of the
parameters but linear in terms of the residuals' squares. 

\textbf{Basic Concept}:

The goal of parameter estimation is to find the set of parameters 
$\theta$ that minimizes the difference between the observed data and the model's predictions (the residual described as following)
y is the date set from experiments and f is the prediction from the nonlinear model.

\textbf{Advantages and Limitations}
\textbf{Advantages}:

Efficiency: The Gauss-Newton algorithm is more efficient than general non-linear optimization methods because it exploits the problem structure.
Simplicity: The update step is relatively simple, especially when the Jacobian can be easily computed.

\textbf{Limitations}:

Convergence: The algorithm may not converge if the initial guess is too far from the true solution, or if the problem is poorly conditioned.
Non-convexity: In the presence of multiple local minima, the algorithm might converge to a local, rather than global, minimum.

\end{document}