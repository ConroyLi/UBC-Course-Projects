\documentclass[a4paper,12pt]{article} % The document class with options

\usepackage[T1]{fontenc}
\usepackage{amsmath}
\usepackage{amsfonts}
\usepackage{microtype}
\usepackage{graphicx}
\usepackage{epstopdf}
% chktex-file 1
% chktex-file 3
% chktex-file 8
% chktex-file 13
% chktex-file 18
% chktex-file 24
% chktex-file 44

\begin{document}
\setlength{\parskip}{1em} 
\setlength{\parindent}{0pt}
\newcommand{\vect}[1]{\mathbf{#1}}

\title{MECH 570C-FSI: Coding Project 2
Fluid-structure interaction with nonlinear hyperleastic structure}
\author{Jincong Li \\ 60539939}
\date{Mar 26th}
\maketitle

\section*{Abstract}
This project extends the exploration of Fluid-Structure Interaction (FSI) 
to encompass the dynamics of hyperelastic materials in fluid flows, advancing 
from the simpler interactions of rigid cylindrical structures. Our focus 
transitions to simulating the complex behavior of a hyperelastic flag 
subjected to fluid dynamics, employing a comprehensive three-dimensional 
nonlinear structural model. By integrating the St. Venant-Kirchhoff material 
model, characterized by Lamé coefficients, Young's modulus, and Poisson's 
ratio, we detail the structural displacements within a Lagrangian framework. 
The methodology involves discretizing the governing equations using finite 
element analysis, coding the resulting model in MATLAB as "hyperelasticMaterial.m", 
and employing the Newton-Raphson technique for iterative solutions. Initial 
validation is conducted through isolated structural tests based on the 
Turek cylinder-bar benchmark, progressing to the integration of fluid 
forces and the subsequent application to a flapping filament problem at 
specified Reynolds numbers. Comparative analysis with established benchmarks 
underscores the solver's efficacy and fidelity in replicating the nuanced 
physics of FSI in hyperelastic structures. This project not only broadens 
the understanding of hyperelastic material dynamics in fluid flows but 
also lays foundational work for advanced simulations in the realm of fluid-structure interactions.

\section*{Introduction}
The genesis of this project lies in the quest to extend our comprehension and simulation 
capabilities beyond rigid cylindrical structures interacting with fluid flows, 
as explored in our preliminary assignment. The focal point shifts towards the 
modeling and simulation of a flexible hyperelastic structure, such as a flag, 
subject to the dynamic forces of a surrounding fluid. This progression is not 
merely academic but a necessary leap towards capturing the real-world complexities 
encountered in engineering and design. Hyperelastic materials, characterized by 
their ability to undergo large deformations and return to their original state, 
necessitate a nonlinear structural analysis to accurately predict their behavior under load.

\section*{Methodology}
The methodology adopted in this project is a multi-step process beginning with the 
discretization of the governing structural equations using the finite element method 
(FEM). This approach enables the approximation of the equations' solutions over a 
discretized model of the structure. Following the discretization, the nonlinear 
equations are coded into a MATLAB script, hyperelasticMaterial.m, akin to the 
format used in our initial exploration of Navier-Stokes equations for fluid flow 
simulation. Critical to solving the nonlinear structural equations is the Newton-Raphson 
technique, renowned for its efficiency in handling nonlinear systems through iterative approximation.

\end{document}